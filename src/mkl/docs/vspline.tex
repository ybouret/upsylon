\documentclass[aps,12pt]{revtex4}
\usepackage{graphicx}
\usepackage{amssymb,amsfonts,amsmath,amsthm}
\usepackage{chemarr}
\usepackage{bm}
\usepackage{pslatex}
\usepackage{bookman}



\begin{document}
\title{Virtual Splines}
\maketitle
	

\section{Description}
We have a set of $N$ points $\vec{P}_1,\ldots,\vec{P}_N$, each point in $\mathbb{R}^d$ (where $d$ is the space dimension), and we assume
that they describe a virtual arc for a dimensionless coordinate $t_i=i$.
We define:
\begin{equation}
 	\begin{array}{rcl|rcl|rcl}
	B(\tau) & = & \tau  & B'(\tau) & = & 1 & B''(\tau) & = & 0\\
	A(\tau) & = & B(1-\tau) & A'(\tau) & = & -1 & A''(\tau) & = & 0\\
	D(\tau) & = & \dfrac{\tau^3-\tau}{6} & D'(\tau) & = & \dfrac{3\tau^2-1}{6}& D'(\tau) & = & \tau\\
	C(\tau) & = & D(1-\tau) & C'(\tau) & = & \dfrac{1-3(1-\tau)^2}{6} & C''(\tau) & = & 1-\tau\\
	F(\tau) & = & \dfrac{3\tau^5-5\tau^4+2\tau}{60} & F'(\tau) & = & ...& F''(\tau) & = & \tau^3-\tau^2\\
	E(\tau) & = & F(1-\tau) & E'(\tau) & = & -F'(1-\tau) & E''(\tau) & = & (1-\tau)^3-(1-\tau)^2\\
	\end{array}
 \end{equation}
 We get:
\begin{itemize}
\item
\begin{equation}
\vec{P}(\tau) = A(\tau)\vec{P}_i + B(\tau) \vec{P}_{i+1} + C(\tau)\vec{P}_{i}'' + D(\tau) \vec{P}_{i+1}''
\end{equation}
which go through all the points and second derivatives,
\item
\begin{equation}
\vec{Q}(\tau) = A(\tau)\vec{P}_i + B(\tau) \vec{P}_{i+1} + C(\tau)\vec{P}_{i}'' + D(\tau) \vec{P}_{i+1}'' + G(\tau) \left[ \vec{P}_{i+1}'' - \vec{P}_{i}''\right]
\end{equation}
which go through all the points and second derivative, with a null third derivative for each points.
\end{itemize}

\section{Continuity}

\subsection{For $\vec{P}$}

\subsubsection{Bulk}

\begin{equation}
\left\lbrace
\begin{array}{rcl}
	\vec{P}'(0) & = & \left[\vec{P}_{i+1}-\vec{P}_i\right] - \dfrac{1}{6} \vec{A}_{i+1} - \dfrac{1}{3}\vec{A}_{i} \\
	\\
	\vec{P}'(1) & = & \left[\vec{P}_{i+1}-\vec{P}_i\right] + \dfrac{1}{3} \vec{A}_{i+1} + \dfrac{1}{6}\vec{A}_{i}\\
\end{array}
\right.
\end{equation}
and
\begin{equation}
	 \left[\vec{P}_{i+1}-\vec{P}_i\right] - \dfrac{1}{6} \vec{A}_{i+1} - \dfrac{1}{3}\vec{A}_{i}
	 =
	 \left[\vec{P}_{i}-\vec{P}_{i-1}\right] + \dfrac{1}{3} \vec{A}_{i} + \dfrac{1}{6}\vec{A}_{i-1}
\end{equation}
then
\begin{equation}
	 \left[\vec{P}_{i+1}-2\vec{P}_i+\vec{P}_{i-1}\right] = \dfrac{1}{6} \vec{A}_{i-1} + \dfrac{2}{3}\vec{A}_{i} + \dfrac{1}{6} \vec{A}_{i+1}
\end{equation}

\subsubsection{Fixed Boundaries}
 
\begin{equation}
\left\lbrace
\begin{array}{rcl}
 	 \left[\vec{P}_{2}-\vec{P}_1\right] - \vec{P}'_{lower}& = &\dfrac{1}{3} \vec{A}_1 + \dfrac{1}{6} \vec{A}_{2}\\
	 \\
	\vec{P}'_{upper} - \left[\vec{P}_N - \vec{P}_{N-1}\right] & = & \dfrac{1}{6} \vec{A}_{N-1} + \dfrac{1}{3} \vec{A}_{N}\\
\end{array}
\right.
\end{equation}
 
 
\subsection{For $\vec{Q}$}
\subsubsection{Bulk}

\begin{equation}
\left\lbrace
\begin{array}{rcl}
	\vec{Q}'(0) & = & \left[\vec{P}_{i+1}-\vec{P}_i\right] - \dfrac{3}{20} \vec{A}_{i+1} - \dfrac{7}{20}\vec{A}_{i} \\
	\\
	\vec{Q}'(1) & = & \left[\vec{P}_{i+1}-\vec{P}_i\right] + \dfrac{7}{20} \vec{A}_{i+1} + \dfrac{3}{20}\vec{A}_{i}\\
\end{array}
\right.
\end{equation}
then:
\begin{equation}
	 \left[\vec{P}_{i+1}-2\vec{P}_i+\vec{P}_{i-1}\right] = \dfrac{3}{20} \vec{A}_{i-1} + \dfrac{7}{10}\vec{A}_{i} + \dfrac{3}{20} \vec{A}_{i+1}
\end{equation}

\subsubsection{Fixed Boundaries}
 
\begin{equation}
\left\lbrace
\begin{array}{rcl}
 	 \left[\vec{P}_{2}-\vec{P}_1\right] - \vec{P}'_{lower}& = &\dfrac{7}{20} \vec{A}_1 + \dfrac{3}{20} \vec{A}_{2}\\
	 \\
	\vec{P}'_{upper} - \left[\vec{P}_N - \vec{P}_{N-1}\right] & = & \dfrac{3}{20} \vec{A}_{N-1} + \dfrac{7}{20} \vec{A}_{N}\\
\end{array}
\right.
\end{equation}


\end{document}
