\documentclass[aps,12pt]{revtex4}
\usepackage{graphicx}
\usepackage{amssymb,amsfonts,amsmath,amsthm}
\usepackage{chemarr}
\usepackage{bm}
\usepackage{pslatex}
\usepackage{bookman}



\begin{document}
\title{Virtual Splines}
\maketitle
	
We have a set of $N$ points $\vec{P}_1,\ldots,\vec{P}_N$, each point in $\mathbb{R}^d$ (where $d$ is the space dimension), and we assume
that they describe a virtual arc for a dimensionless coordinate $t_i=i$.
 

\section{Regular V-Spline}
\subsection{Coefficients}
\begin{equation}
\forall \tau \in [0:1], \;
\left\lbrace
\begin{array}{rcl}
\alpha & = & 1-\tau\\
\beta  & = & \tau\\
\gamma & = & \dfrac{ (1-\tau)^3 - (1-\tau) }{6}\\
\delta & = & \dfrac{\tau^3-\tau}{6}\\
\end{array}
\right.
\Rightarrow
\left\lbrace
\begin{array}{rcl}
\alpha' & = & -1\\
\beta'  & = & 1\\
\gamma' & = & \dfrac{ 1-3(1-\tau)^2 }{6}\\
\delta' & = & \dfrac{3\tau^2-1}{6}\\
\end{array}
\right.
\Rightarrow
\left\lbrace
\begin{array}{rcl}
\alpha'' & = & 0\\
\beta''  & = & 0\\
\gamma'' & = & 1-\tau\\
\delta'' & = & \tau\\
\end{array}
\right.
\end{equation}

\begin{equation}
\left\lbrace
\begin{array}{rcl}
\vec{P}(\tau)  & = & \alpha(\tau)\vec{P}_i + \beta(\tau) \vec{P}_{i+1} + \gamma(\tau)\vec{P}_{i}'' + \delta(\tau) \vec{P}_{i+1}''\\
\\
\vec{P}'(\tau) & = & \left[\vec{P}_{i+1} - \vec{P}_i\right] + \dfrac{1}{6} \left[ \underbrace{(1-3(1-\tau)^2)}_{(6-3\tau)\tau-2} \vec{P}_{i}'' + (3\tau^2-1) \vec{P}_{i+1}'' \right] \\
\\
\vec{P}''(\tau) & = & (1-\tau) \vec{P}_{i}'' + \tau \vec{P}_{i+1}''\\
\\
\vec{P}'''(\tau) & = & \vec{P}_{i+1}'' - \vec{P}_i''\\
\end{array}
\right.
\end{equation}

\subsection{Continuity}

\subsubsection{Bulk}

\begin{equation}
\left\lbrace
\begin{array}{rcl}
	\vec{P}'(0) & = & \left[\vec{P}_{i+1}-\vec{P}_i\right] - \dfrac{1}{6} \vec{A}_{i+1} - \dfrac{1}{3}\vec{A}_{i} \\
	\\
	\vec{P}'(1) & = & \left[\vec{P}_{i+1}-\vec{P}_i\right] + \dfrac{1}{3} \vec{A}_{i+1} + \dfrac{1}{6}\vec{A}_{i}\\
\end{array}
\right.
\end{equation}
and
\begin{equation}
	 \left[\vec{P}_{i+1}-\vec{P}_i\right] - \dfrac{1}{6} \vec{A}_{i+1} - \dfrac{1}{3}\vec{A}_{i}
	 =
	 \left[\vec{P}_{i}-\vec{P}_{i-1}\right] + \dfrac{1}{3} \vec{A}_{i} + \dfrac{1}{6}\vec{A}_{i-1}
\end{equation}
then
\begin{equation}
	 \left[\vec{P}_{i+1}-2\vec{P}_i+\vec{P}_{i-1}\right] = \dfrac{1}{6} \vec{A}_{i-1} + \dfrac{2}{3}\vec{A}_{i} + \dfrac{1}{6} \vec{A}_{i+1}
\end{equation}

\subsubsection{Fixed Boundaries}
 
\begin{equation}
\left\lbrace
\begin{array}{rcl}
 	 \left[\vec{P}_{2}-\vec{P}_1\right] - \vec{P}'_{lower}& = &\dfrac{1}{3} \vec{A}_1 + \dfrac{1}{6} \vec{A}_{2}\\
	 \\
	\vec{P}'_{upper} - \left[\vec{P}_N - \vec{P}_{N-1}\right] & = & \dfrac{1}{6} \vec{A}_{N-1} + \dfrac{1}{3} \vec{A}_{N}\\
\end{array}
\right.
\end{equation}

\section{Other V-Spline}
\subsection{Coefficients}
We use:
\begin{equation}
	\eta(\tau) = \dfrac{\tau}{60} - \dfrac{\tau^3}{6} + \dfrac{\tau^4}{4} - \dfrac{\tau^5}{10}
	 = \dfrac{1}{30} \left(\dfrac{1}{2} - \tau \right)\tau (1-\tau) \left[ 1 + 3 \tau(1-\tau) \right]
\end{equation}
with
\begin{equation}
\eta'(\tau) = \dfrac{1}{60} - \dfrac{1}{2} \left[ \tau(1-\tau) \right]^2
\end{equation}

\begin{equation}
\eta''(\tau) =  \tau(1-\tau)(2\tau-1)
\end{equation}

\begin{equation}
\eta'''(\tau) =  6\tau(1-\tau)-1
\end{equation}

\end{document}

\subsection{For $\vec{Q}$}
\subsubsection{Bulk}

\begin{equation}
\left\lbrace
\begin{array}{rcl}
	\vec{Q}'(0) & = & \left[\vec{P}_{i+1}-\vec{P}_i\right] - \dfrac{3}{20} \vec{A}_{i+1} - \dfrac{7}{20}\vec{A}_{i} \\
	\\
	\vec{Q}'(1) & = & \left[\vec{P}_{i+1}-\vec{P}_i\right] + \dfrac{7}{20} \vec{A}_{i+1} + \dfrac{3}{20}\vec{A}_{i}\\
\end{array}
\right.
\end{equation}
then:
\begin{equation}
	 \left[\vec{P}_{i+1}-2\vec{P}_i+\vec{P}_{i-1}\right] = \dfrac{3}{20} \vec{A}_{i-1} + \dfrac{7}{10}\vec{A}_{i} + \dfrac{3}{20} \vec{A}_{i+1}
\end{equation}

\subsubsection{Fixed Boundaries}
 
\begin{equation}
\left\lbrace
\begin{array}{rcl}
 	 \left[\vec{P}_{2}-\vec{P}_1\right] - \vec{P}'_{lower}& = &\dfrac{7}{20} \vec{A}_1 + \dfrac{3}{20} \vec{A}_{2}\\
	 \\
	\vec{P}'_{upper} - \left[\vec{P}_N - \vec{P}_{N-1}\right] & = & \dfrac{3}{20} \vec{A}_{N-1} + \dfrac{7}{20} \vec{A}_{N}\\
\end{array}
\right.
\end{equation}


\end{document}
