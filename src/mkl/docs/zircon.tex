\documentclass[aps,12pt]{revtex4}
\usepackage{graphicx}
\usepackage{amssymb,amsfonts,amsmath,amsthm}
\usepackage{chemarr}
\usepackage{bm}
\usepackage{pslatex}
\usepackage{mathptmx}
\usepackage{xfrac}
\usepackage{bookman}

%%  
\newcommand{\trn}[1]{{#1}^{\mathtt{T}}}
 
\begin{document}

We define $\vec{F} : \mathbb{R}^N \rightarrow \mathbb{R}^N$ and we want to solve $\vec{F}(\vec{X})=\vec{0}$.
Starting from $\vec{X}_n$:
\begin{equation}
	\vec{F}(\vec{X} + \vec{h}) \simeq \vec{F} + \bm{J} \vec{h}
\end{equation}
We defined the associated problem using:
\begin{equation}
	g(\vec{X}) = \frac{1}{2} \vec{F}^2(\vec{X}) 
\end{equation}
so that:
\begin{equation}
	g(\vec{X}+\vec{h}) \simeq g(\vec{X}) + \vec{\nabla}g \cdot \vec{h} + \frac{1}{2} \trn{\vec{h}} \bm{H} \vec{h}
\end{equation}
with:
\begin{equation}
	\vec{\nabla}g =  \trn{\bm{J}} \vec{F}
\end{equation}
and:
\begin{equation}
	\bm{H}_{ij} = \dfrac{\partial^2 g}{\partial_{X_i}\partial_{X_j}} = \partial_{X_i} \vec{F} \cdot \partial_{X_j} \vec{F} + \vec{F} \cdot \partial_{X_iX_j} \vec{F}
\end{equation}
With $\vec{F}\approx\vec0$,
\begin{equation}
	\bm{H} \simeq \bm{C}  = \trn{\bm{J}} \bm{J}
\end{equation}
Finally, we want to minimize towards $0$:
\begin{equation}
	g(\vec{X} +\vec{h}) \simeq g(\vec{X}) + \trn{\vec{F}} \bm{J}  \vec{h} + \frac{1}{2} \trn{\vec{h}} \bm{C} (\lambda) \vec{h}
\end{equation}
where $\bm{C}(\lambda)$ is a regularised matrix such that $\bm{C}(0)=\trn{\bm{J}}\bm{J}$.
A solution is
\begin{equation}
	\vec{h} = -\vec{C}^{-1}(\lambda) \trn{\bm{J}} \vec{F}
\end{equation}

\end{document}
