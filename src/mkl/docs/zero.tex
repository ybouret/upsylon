\documentclass[aps,12pt]{revtex4}
\usepackage{graphicx}
\usepackage{amssymb,amsfonts,amsmath,amsthm}
\usepackage{chemarr}
\usepackage{bm}
\usepackage{pslatex}
\usepackage{mathptmx}
\usepackage{xfrac}

%%  
 
\begin{document}
We have a continuous function $f(x)$ on $[a;c]$ such that:
 $$f(a)f(c)<0.$$
Let us define:
$$
	b = \dfrac{a+b}{2}
$$
We define $h(x)=f(x)e^{\alpha(x-b)}$ that linearizes $f$ on the interval.
$$
\left\lbrace
\begin{array}{rcl}
	h(a) & = & f(a)e^{\alpha(a-b)}\\
	h(b) & = & f(b)\\
	h(c) & = & f(c) e^{\alpha(c-b)}\\
	h(b) & = & \dfrac{h(a)+h(b)}{2}\\
\end{array}
\right.
$$
We define
$$
Q = e^{\alpha\frac{c-a}{2}}
$$
to get
$$	
\left\lbrace
\begin{array}{rcl}
	h(a) & = & f(a)/Q\\
 	h(c) & = & f(c)Q\\
\end{array}
\right.
$$
We need to solve:
$$
	2f(b) = f(a)/Q+f(c)Q \;\; \Leftrightarrow \;\; f(c)Q^2 - 2 f(b) Q + f(a) = 0
$$
We find:
$$
	\Delta' = f(b)^2 - f(a)f(b), \;\; Q = \dfrac{f(b)+\epsilon_c \sqrt{\Delta'}}{f(c)} = \dfrac{f(a)}{f(b)-\epsilon_c \sqrt{\Delta'}}
$$
so that
$$
\left\lbrace
\begin{array}{rcl}
	f(a)/Q & = & f(b) - \epsilon_c \sqrt{\Delta'}\\
	f(c) Q & = & f(b) + \epsilon_c \sqrt{\Delta'}\\
\end{array}
\right.
$$
Since the three $h$ values are aligned with $h(a)h(c)=f(a)f(b)$, we express the secant value from $b$:
$$
	h(x) \simeq h(b) + \dfrac{x-b}{a-b} \left( h(a) - h(b) \right)
$$
leading to:
$$
\left\lbrace
\begin{array}{rcl}
	x' & = & b - (a-b) \dfrac{h(b)}{h(a)-h(b)} \\
	\\
	& = & b + \dfrac{w}{2} \dfrac{f(b)}{\frac{f(a)}{Q}-f(b)}\\
	\\
	& = & b - \dfrac{w}{2} \dfrac{\epsilon_c f(b)}{\sqrt{\Delta'}}\\
\end{array}
\right.
$$

\end{document}