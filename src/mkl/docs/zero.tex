\documentclass[aps,12pt]{revtex4}
\usepackage{graphicx}
\usepackage{amssymb,amsfonts,amsmath,amsthm}
\usepackage{chemarr}
\usepackage{bm}
\usepackage{pslatex}
\usepackage{mathptmx}
\usepackage{xfrac}

%%  
 
\begin{document}

We have a continuous function $f(x)$ on $[a;b]$ such that:
 $$f(a)f(c)<0.$$
We define $h(x)=f(x)P(x)$ that linearizes $f$ on the interval, namely:
$$b=\frac{a+c}{2},\;\;h(b) = \frac{h(a)+h(c)}{2}.$$
Then an approximation of $x_z$ such that $f(x_z)=0$ if the linear interpolation to $0$
of $h$ on $[a;b]$.\\
We use Ridder's idea to define:
$$P(x)=e^{\alpha(x-a)},$$
leading to:
$$
	2f(b)e^{(b-a)} = f(a) + f(c)e^{(c-a)}
$$
and using:
$$
	Q = e^{(b-a)}
$$
we need to solve:
$$
	2f(b)Q = f(a) + f(c) Q^2.
$$
\end{document}