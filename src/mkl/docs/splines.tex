\documentclass[aps,onecolumn]{revtex4}
\usepackage{graphicx}
\usepackage{amssymb,amsfonts,amsmath,amsthm}
\usepackage{bm}
\usepackage{pslatex}
\usepackage{bookman}

\begin{document}

\title{Splines}
\maketitle

\section{Expression}

Let us define an arc by a set of $N$ points of $\mathbb{R}^d$ at times $t_1<\ldots<t_i<\ldots<t_N$:
\begin{equation}
	\forall i \in [1:N],\;	(t_i,\vec{P}_i).
\end{equation}
For each time $t_i$, we assume that the arc has a velocity $\vec{V}_i$ and and acceleration $\vec{A}_i$.
We define:
\begin{equation}
\left\lbrace
\begin{array}{rcl}
	\delta_i & = & t_{i+1}-t_{i}\\
	\\
	\tau     & = & \dfrac{t-t_{i}}{\delta_i}\\
	\\
	1-\tau   & = & \dfrac{t_{i+1}-t}{\delta_i}\\
\end{array}
\right.
\end{equation}
		
We get the local arc:
\begin{equation}
	\vec{Q}_i(t) = (1-\tau) \vec{P}_i + \tau \vec{P}_{i+1} + \dfrac{ (1-\tau)^3 - (1-\tau) }{6} \vec{A}_{i} + \dfrac{\tau^3-\tau}{6} \vec{A}_{i+1}
\end{equation}	

\begin{equation}
	\dot{\vec{Q}}_i(t) = \dfrac{1}{\delta_i} \left[ \vec{P}_{i+1} - \vec{P}_i + \dfrac{1-3(1-\tau)^2}{6}\vec{A}_i + \dfrac{3\tau^2-1}{6} \vec{A}_{i+1}\right]
\end{equation}

\begin{equation}
	\ddot{\vec{Q}}_i(t) = \dfrac{1}{\delta_i^2} \left[ (1-\tau) \vec{A}_i + \tau \vec{A}_{i+1} \right]
\end{equation}

\section{Continuity}

\subsection{Bulk}
\begin{equation}
\dot{\vec{Q}}_{i-1}(\tau=1) =  \dot{\vec{Q}}_i(\tau=0) \Leftrightarrow 
\dfrac{1}{\delta_{i-1}} \left[ \vec{P}_{i} - \vec{P}_{i-1} + \dfrac{1}{6}\vec{A}_{i-1} + \dfrac{1}{3} \vec{A}_{i}\right]
 = 
\dfrac{1}{\delta_i} \left[ \vec{P}_{i+1} - \vec{P}_i - \dfrac{1}{3}\vec{A}_i - \dfrac{1}{6} \vec{A}_{i+1}\right]
\end{equation}
Leading to:
\begin{equation}
	\dfrac{\delta_i}{6} \vec{A}_{i-1} + \dfrac{1}{3} \left(\delta_i + \delta_{i-1} \right) \vec{A}_i + \dfrac{\delta_{i-1}}{6} \vec{A}_{i+1} =
	\delta_{i-1} \left[ \vec{P}_{i+1} - \vec{P}_i \right] - \delta_i \left[ \vec{P}_i - \vec{P}_{i-1}\right ]
\end{equation}

\subsection{Boundaries}

\subsubsection{Periodic}
\begin{equation}
\dot{\vec{Q}}_{N-1}(\tau=1) =  \dot{\vec{Q}}_1(\tau=0)
\end{equation}

\subsubsection{Standard}

\begin{itemize}
\item Natural:
	\begin{itemize}
	\item at $t_1$ : $\vec{A}_1=\vec{0}$,
	\item at $t_N$ : $\vec{A}_N=\vec{0}$.
	\end{itemize}
\item Fixed:
	\begin{itemize}
	\item at $t_1$ : $\dot{\vec{Q}}_1(\tau=0) = \vec{V}_1$,
	\item at $t_N$ : $\dot{\vec{Q}}_{N-1}(\tau=1) = \vec{V}_N$.
	\end{itemize}	
	
\end{itemize}

\subsection{Computation}
We obtain a tridiagonal matrix for the standard case and a cyclic tridiagonal matrix for the periodic case.
The matrix is the same for each dimension, and the right hand side is to be computed for each dimension.

\section{Evaluation}
\subsection{Standard}


\subsection{Periodic}



\end{document}