\documentclass[aps,12pt]{revtex4}
\usepackage{graphicx}
\usepackage{amssymb,amsfonts,amsmath,amsthm}
\usepackage{chemarr}
\usepackage{bm}
\usepackage{pslatex}
\usepackage{mathptmx}
\usepackage{xfrac}
\usepackage{bookman}

%%  
\newcommand{\trn}[1]{{#1}^{\mathtt{t}}}
 
\begin{document}

Let us assume that $\vec{F}(\vec{X})$ is derivable at $\vec{X}$.
Since:
\begin{equation}
	\vec{F}(\vec{X}+\delta\vec{X}) \simeq \vec{F}(\vec{X}) + \bm{J} \delta\vec{X}
\end{equation}
the full Newton's step is:
\begin{equation}
	\vec{p} = - \bm{J}^{-1} \vec{F}
\end{equation}
We define
\begin{equation}
	g(\lambda) = \dfrac{1}{2} \trn{\vec{F}}  \vec{F} (\vec{X}+\lambda \vec{p})
\end{equation}

\begin{equation}
	\vec{\nabla}g = \trn{J}\vec{F}
\end{equation}
and we want
\begin{equation}
	g(\lambda) \leq g_0 + \alpha \lambda \underbrace{\vec{\nabla}g_0 \cdot \vec{p}}_{-2g_0^2}
\end{equation}

\end{document}