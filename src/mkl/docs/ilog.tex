\documentclass[aps]{revtex4}
\usepackage{graphicx}
\usepackage{amssymb,amsfonts,amsmath,amsthm}
\usepackage{chemarr}
\usepackage{bm}
\usepackage{pslatex}
\usepackage{mathptmx}
\usepackage{xfrac}

%% concentration notations
\newcommand{\mymat}[1]{\boldsymbol{#1}}
\newcommand{\mytrn}[1]{~^{\mathsf{t}}\!{#1}}
\newcommand{\myvec}[1]{\overrightarrow{#1}}
\newcommand{\mygrad}{\vec{\nabla}}
\newcommand{\myhess}{\mathcal{H}}

\newcommand{\myfloor}[1]{{\left\lfloor#1\right\rfloor}}
\newcommand{\myceil}[1]{{\left\lceil#1\right\rceil}}


\begin{document}
\title{Integer Logarithm Approximation}

\section{Bracketing}

We have 
\begin{equation}
	2^p \leq x < 2^{p+1}
\end{equation}
so that
\begin{equation}
	p\ln(2) \leq \log(x) < (p+1) \ln(2)
\end{equation}

\section{Minimisation}
We have
\begin{equation}
\forall x \in [a:b], \;\; \ln(x) \geq \ln(a) + \dfrac{x-a}{b-a} \left( \ln(b) - \ln(a) \right)
\end{equation}
so that
\begin{equation}
\forall x \in [2^p:2^{p+1}],  \;\; \ln(x) \geq \ln(2) \left[ p + \left( \dfrac{x}{2^p}-1 \right) \right]
\end{equation}

\section{Maximisation}

\begin{equation}
	f(x) = \ln(a) + \dfrac{1}{a}  (x-a) + \alpha (x-a)^2 + \beta (x-a)^3
\end{equation}

\begin{equation}
	 f'(x) = \dfrac{1}{a} + 2 \alpha (x-a) + 3 \beta (x-a)^2
\end{equation}

\begin{equation}
\left\lbrace
	\begin{array}{rcl}
	\ln(b) & = & \ln(a) + \dfrac{1}{a} (b-a) + \alpha (b-a)^2 + \beta (b-a)^3\\
	\\
	\dfrac{1}{b} & = & \dfrac{1}{a} + 2\alpha(b-a) + 3\beta(b-a)^2\\
	\end{array}
\right.
\end{equation}

\begin{equation}
\left\lbrace
	\begin{array}{rcl}
	\alpha (b-a) + \beta (b-a)^2 & = & \dfrac{\ln(b) - \ln(a)}{b-a} - \dfrac{1}{a}\\
	\\
	2\alpha(b-a)  + 3\beta(b-a)^2  & = & \dfrac{1}{a}-\dfrac{1}{b}  \\
	\end{array}
\right.
\end{equation}



\end{document}


