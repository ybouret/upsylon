\documentclass[aps,10pt]{revtex4}
\usepackage{graphicx}
\usepackage{amssymb,amsfonts,amsmath,amsthm}
\usepackage{chemarr}
\usepackage{bm}
\usepackage{pslatex}
\usepackage{bookman}

\newcommand{\myvec}[1]{\overrightarrow{#1}}
%\newcommand{\mynorm}[1]{\left\vert#1\right\vert}
\newcommand{\mynorm}[1]{{\parallel{#1}\parallel}}
\newcommand{\myunit}[1]{\dfrac{{#1}}{\mynorm{{#1}}}}

\begin{document}
\title{Local Curve Approximations}
\maketitle

\section{Description}

Let us assume that we have three different points $\vec{A}$, $\vec{B}$, $\vec{C}$.

\section{Parametrization}

\subsection{Bulk Expression}
\subsubsection{2D/3D Case}
\begin{equation}
	\vec{M} = \vec{B} + x \vec{V} + \dfrac{1}{2}x^2 \vec{W}
\end{equation}
leading to
\begin{equation}
\left\lbrace
\begin{array}{rcl}
	\vec{A} & = & \vec{B} - \vec{V} + \dfrac{1}{2}\vec{W}\\
	\\
	\vec{C} & = & \vec{B} + \vec{V} + \dfrac{1}{2}\vec{W}\\
\end{array}
\right.
\Rightarrow
\left\lbrace
\begin{array}{rcl}
	\vec{V} & = & \dfrac{1}{2}\left(\vec{C}-\vec{A}\right)\\
	\\
	\vec{W} & = &  \left(\vec{A}+\vec{C}-2\vec{B}\right)\\
\end{array}
\right.
\end{equation}
so that the tangent vector is:
\begin{equation}
	\vec{T}_B = \myunit{\vec{V}} = \myunit{\vec{C}-\vec{A}}
\end{equation}

\subsubsection{In 2D}
\begin{equation}
	\gamma = \dfrac{\det\left(\vec{V},\vec{W}\right)}{\mynorm{\vec{V}}^3} = \dfrac{4\det\left(\vec{C}-\vec{A},\vec{A}+\vec{C}-2\vec{B}\right)}{\mynorm{\vec{C}-\vec{A}}^3}
\end{equation}

\begin{equation}
	\vec{N}_{B} = \mathrm{rotate}_{\frac{\pi}{2}} \left(\vec{T}_B\right)
\end{equation}

\subsubsection{In 3D}

\begin{equation}
	\gamma = \dfrac{\mynorm{\vec{V}\wedge\vec{W}}}{\mynorm{\vec{V}}^3} = 
	\dfrac{4\mynorm{\left(\vec{C}-\vec{A}\right)\wedge\left(\vec{A}+\vec{C}-2\vec{B}\right)}}
	{\mynorm{\vec{C}-\vec{A}}^3}
\end{equation}

\subsection{Head Expression}

\subsection{Tail Expression}


\section{Unit Tangent Vectors}

\begin{equation}
\left\lbrace
\begin{array}{rcl}
	\vec{T}_A & = & \myunit{4\vec{B}-3\vec{A}-\vec{C}}\\
	\\
	\vec{T}_B & = & \myunit{\vec{C}-\vec{A}}  \\
	\\
	\vec{T}_C & = & \myunit{3\vec{C}+\vec{A}-4\vec{B}}\\
\end{array}
\right.
\end{equation}

\section{Extrapolation}

Let us assume that we now have a segment starting from $\vec{P}$ and going to $\vec{Q}$, with an initial derivative $\vec{S}_P$ and a final derivative $\vec{S}_Q$.
The expression of the interpolating function is
\begin{equation}
	\vec{M} = A \vec{P} + B \vec{Q} + C \vec{U} + D \vec{V}
\end{equation}
with, using $p$ as the index of $\vec{P}$ and $x\in[p,p+1]$,
\begin{equation}
	B = x-p,\; A=1-B,\; C = \frac{1}{6}(A^3-A), \; D=\frac{1}{6}(B^3-B)
\end{equation}
leading to
\begin{equation}
	\partial_x \vec{M} = \overrightarrow{PQ} - \frac{3A^2-1}{6} \vec{U} + \frac{3B^2-1}{6} \vec{V}
\end{equation}
and we got the system
\begin{equation}
\left\lbrace
\begin{array}{rcrcl}
	\vec{S}_P & = & \partial_x\vec{M}|_{(A=1,B=0)} & = & \overrightarrow{PQ} - \dfrac{2}{6}\vec{U} - \dfrac{1}{6}\vec{V} \\
	\\
	\vec{S}_Q & = & \partial_x\vec{M}|_{(A=0,B=1)} & = & \overrightarrow{PQ} + \dfrac{1}{6}\vec{U} + \dfrac{2}{6}\vec{V} \\

\end{array}
\right.
\end{equation}
or
\begin{equation}
\left\lbrace
\begin{array}{rcl}
	6\left[\overrightarrow{PQ} - \vec{S}_P \right] & = & 2\vec{U} + \vec{V}\\
	\\
	6\left[\vec{S}_Q - \overrightarrow{PQ} \right] & = & \vec{U} + 2\vec{V}\\
	\end{array}
\right.
\end{equation}

\begin{equation}
	\begin{bmatrix}
	\vec{U}\\
	\vec{V}\\
	\end{bmatrix}
	= 2 
	\begin{bmatrix}
	2 & -1\\
	-1 & 2\\
	\end{bmatrix}
	\begin{bmatrix}
	 \overrightarrow{PQ} - \vec{S}_P  \\
	 \vec{S}_Q - \overrightarrow{PQ}  \\
	\end{bmatrix}
\end{equation}

\end{document}