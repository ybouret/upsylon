\documentclass[aps,onecolumn,11pt]{revtex4}
\usepackage{graphicx}
\usepackage{amssymb,amsfonts,amsmath,amsthm}
\usepackage{bm}
\usepackage{xfrac}
\usepackage{chemarr}

\usepackage{bookman}

\newcommand{\rank}{\mathrm{rank}}

\begin{document}
\title{\Large Finding Orthogonal Spaces}
\maketitle

Let $\bm{U}\in\mathcal{M}_{N,M}, \; N<M$ with $\rank(\bm{U})=N$.
We want to find $\bm{V}\in\mathcal{M-N,M}$ such that:
\begin{itemize}
\item the rows of
$\bm{V}$ are in $\bm{U}^\perp$,
\item and $\rank(\bm{V})=M-N=N_c$.
\end{itemize}

Let $\vec{A}\in\mathbb{R}^M$, $\vec{x}\in\mathbb{R}^N$ and $\vec{y}\in\mathbb{R}^{N_c}$ such that:
\begin{equation}
	\vec{A} = \bm{U}^t \vec{x} + \bm{V}^t \vec{y}.
\end{equation}
We compute the pseudo-inverse of $\bm{U}$, which is the projection matrix on the $\bm{U}$ space:
\begin{equation}
	\bm{P} = \bm{U}^t \left(\bm{U}\bm{U}^t\right)^{-1} \bm{U}
\end{equation}

We decompose:
\begin{equation}
	\vec{A} = \bm{P} \vec{A} + \underbrace{\left[\bm{I}_M - \bm{P} \right]}_{\bm{Q}} \vec{A}
\end{equation}
Accordingly, the \textit{columns} of $\bm{Q}$ are vectors which are orthogonal to each rows of $\bm{U}$.
So, we deduce that \textit{the rows of $\bm{V}$ are a subset of the rows of $\bm{Q}^t$}.
\\
If $\bm{U}$ has integer coefficients, the use of the adjoint matrices allows the build up of
a matrix $\bm{V}$ with integer coefficients.
\end{document}