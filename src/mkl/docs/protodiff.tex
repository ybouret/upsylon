\documentclass[aps,12pt]{revtex4}
\usepackage{graphicx}
\usepackage{amssymb,amsfonts,amsmath,amsthm}
\usepackage{chemarr}
\usepackage{bm}
\usepackage{pslatex}
\usepackage{bookman}



\begin{document}
\title{Diffusion of protic species within confined, aqueous and reactive media}
\maketitle
	
\section{Description}	

\subsection{Notations and framework}
 
\subsubsection{Hypothesis}
We make the following assumptions:
\begin{itemize}
\item we observe some species in a confined medium where no convection occurs,
\item in this medium, the ionic strength is high enough to allow the sole autodiffusion of each ion, without further electrostatic coupling,
\end{itemize}


\subsubsection{Algebraic problem}
Consequently, each species $S$ :
\begin{itemize}
\item follow the second Fick law with its autodiffusion coefficient $D_S$:
\begin{equation}
\label{eq:fick}
	\partial_t [S] = D_S \Delta [S],
\end{equation}

\item and obeys the setup boundary conditions.
\end{itemize}

Moreover, each species may be involved in one or many reactions that shall produce some additional terms in \eqref{eq:fick}.

\subsubsection{Space and time scales}

Let us consider the following handbook table extract:
\begin{equation}
\label{tab:diff}
\begin{array}{|c|c|}
\hline
\text{species} & \text{diffusion coefficient in } \text{m}^2\text{s}^{-1} \\
\hline
\hline
H^+ & 9.31 \cdot 10^{-9} \\
HO^- & 5.27 \cdot 10^{-9} \\
CH_3COO^- & 1.09 \cdot 10^{-9}\\
\hline
\end{array}
\end{equation}

Let us consider the following equilibrium:
\begin{equation}
\label{eq:water}
H_2O \xrightleftharpoons[]{} H^+ + HO^-, \;\; K_w.
\end{equation}
The relaxation time for water self-dissociation is around a microsecond, and this is the characteristic time-scale
for the protic equilibria (see \cite{PLOS} for a short review).
Accordingly, for length scales above a few nanometers, \textbf{the diffusion is a perturbation of all the protic equilibria}.

\section{Asymptotic diffusion}

\subsection{Specific problem}



\section{References}
\begin{thebibliography}{1}

\bibitem[Bouret14]{PLOS}
	Y.Bouret,M.Argentina and L.Counillon
	\textit{Capturing Intracellular pH Dynamics by Coupling Its Molecular Mechanisms within a Fully Tractable Mathematical Model}
	PLOS One 2014
	https://doi.org/10.1371/journal.pone.0085449

\end{thebibliography}

\end{document}
