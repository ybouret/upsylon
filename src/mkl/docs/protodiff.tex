\documentclass[aps,11pt]{revtex4}
\usepackage{graphicx}
\usepackage{amssymb,amsfonts,amsmath,amsthm}
\usepackage{chemarr}
\usepackage{bm}
\usepackage{pslatex}
\usepackage{bookman}



\begin{document}
\title{Diffusion of protic species within confined, aqueous and reactive media}
\maketitle
	
\section{Description and framework}	

\subsection{Hypothesis}
 
We make the following assumptions:
\begin{itemize}
\item we observe some species in a confined medium where no convection occurs,
\item in this medium, the ionic strength is high enough to allow the sole autodiffusion of each ion, without further electrostatic coupling,
\end{itemize}


Consequently, :
\begin{itemize}
\item each species $S$ follow the second Fick law with its autodiffusion coefficient $D_S$:
\begin{equation}
\label{eq:fick}
	\partial_t [S] = D_S \Delta [S],
\end{equation}

\item and each species obeys the setup geometrical boundary conditions.
\end{itemize}

Moreover, each species may be involved in one or many reactions that shall produce some additional terms in Eq.\eqref{eq:fick}.

\subsection{Chemical time scales}

Let us consider the following equilibria and their constants:
\begin{equation}
\label{eqs}
\left\lbrace
\begin{array}{rl}
H_2O \xrightleftharpoons[]{} H^+ + HO^-, &K_w \\
AH  \xrightleftharpoons[]{} H^+ + A-, & K_a\\
\end{array}
\right.
\end{equation}
The relaxation time for water self-dissociation is around a microsecond, and this is the characteristic time-scale
for about all protic equilibria. Physiologically, this fast relaxation times allow to consider some "slower" reactions (mostly membrane transport of protic species) as perturbation of all the protic equilibria.\\
A short review of these time scales and a way to derive the proper kinetic equations in such a case were derived in \cite{PLOS} and furthermore applied in \cite{BBA}.\\

\subsection{Diffusive time scales}
Let us now consider the following handbook table extract:
\begin{equation}
\label{tab:diff}
\begin{array}{|c|c|}
\hline
\text{species} & \text{diffusion coefficient in } \text{m}^2\cdot\text{s}^{-1} \\
\hline
\hline
H^+     & 9.31 \cdot 10^{-9} \\
HO^-    & 5.27 \cdot 10^{-9} \\
 HEPES=AH  &   0.50 \cdot 10^{-9}\\
~^-EPES=A- &   0.50 \cdot 10^{-9}\\
\hline
\end{array}
\end{equation}
Here, the fastest diffusive species is the proton, according to both its size and the hydrogen bond dynamics.
Consequently, a simple estimation points out that for length scales above a few nanometers, \textit{the diffusion is a perturbation of all the protic equilibria}.
Hence, we will use here and adapted version of the computational framework successfully harnessed in the previous works.

\section{Asymptotic diffusion}

\subsection{Specific chemical problem}
We want to simulate the diffusion in a space described by $\vec{r}$ and with respect to time $t$ 
of four species, $H^+$, $HO^-$, $AH$ and $A^-$, starting from an equilibrium state where everywhere the following conditions are satisfied from Eq.\eqref{eqs}:
\begin{equation}
\label{eq:steady}
\left\lbrace
\begin{array}{rcl}
0 & = & K_w - [H^+]  [HO^-]\\
0 & = & K_a   [AH] - [H^+][A^-] \\
\end{array}
\right.
\end{equation}
We define:
\begin{itemize}
\item the vector of concentrations:
\begin{equation}
\vec{C} = 
\begin{pmatrix}
[H^+]\\
[HO^-]\\
[AH]\\
[A^-]\\
\end{pmatrix},
\end{equation}

\item the extent matrix:
\begin{equation}
	\bm{\nu} = 
	\begin{pmatrix}
	1 & 1 & 0 & 0 \\
	1 & 0 & -1 & 1\\
	\end{pmatrix},
\end{equation}

\item and the Jacobian matrix of Eq.\ref{eq:steady}:
\begin{equation}
	\bm{J} = 
	\begin{pmatrix}
	-[HO^-] & -[H^+] & 0 & 0 \\
	-[A^-] & 0 & K_a & -[H^+]\\
	\end{pmatrix}
\end{equation}

\end{itemize}

\subsection{Specific diffusion problem}

We know need to move the species according to:
\begin{equation}
\label{eq:move}
\left\lbrace
\begin{array}{lcll}
\partial_t [H^+]  & = & D_{H}  &\Delta [H^+]\\
\partial_t [HO^-] & = & D_{OH} &\Delta [HO^-]\\
\partial_t [AH]   & = & D_{AH} &\Delta [AH]\\
\partial_t [A^-]  & = & D_{A}  &\Delta [A-]\\
\end{array}
\right.
\end{equation}
Once the boundary conditions are chosen, we implement an alternating-direction Crank-Nicolson scheme to obtain
the diffusional increase $\delta \vec{C}(\vec{r},t)$ at each chosen point in space, for a time-step $\delta t$, which is formally an operator $\mathcal{L}$:
\begin{equation}
	\delta \vec{C}(\vec{r},t) = \mathcal{L}(\vec{C}(\vec{r},t),\delta t)
\end{equation}
This computation generally requires the knowledge of all the concentration fields, and is therefore not prone to parallelisation. 
     
\subsection{Asymptotic diffusion}
Obviously, the unconstrained step $\delta \vec{C}$ sends $\vec{C} + \delta\vec{C}$ out of the "allowed" conditions of Eq.\eqref{eq:steady}.
As shown in previous works, the chemical system produces an instantaneous extent (namely very fast compared to the diffusive perturbation) which
modifies the concentrations to return to a "legal" state. The constrained value is:
\begin{equation}
\label{eq:chi}
	\delta_\chi \vec{C}(\vec{r},t) = \underbrace{\left[\bm{I} - ~^t\bm{J}\left(\bm{J}~^t\bm{\nu}\right)^{-1} \bm{J}\right]}_{\text{of } \vec{C}(\vec{r},t)} 
	\mathcal{L}(\vec{C}(\vec{r},t),\delta t)
\end{equation}
For each point in space, once $\mathcal{L}(\vec{C}(\vec{r},t),\delta t)$ is computed, this is a local transform which offers the possibility to be carried out in parallel.\\
Eventually, we define by Eq.\eqref{eq:chi} the modified diffusion step, \textit{where an asymptotically fast relaxing system is driven by a perturbative diffusion}.
\subsection{Chemical balance}
From now on, we compute the "asymptotic" values of the new concentrations:
\begin{equation}
\vec{C}_\chi(\vec{r},t+\delta t) = \vec{C}(\vec{r},t) + \left[\bm{I} - ~^t\bm{J}\left(\bm{J}~^t\bm{\nu}\right)^{-1} \bm{J}\right] \mathcal{L}(\vec{C},\delta t)
\end{equation}  
But since the constraints defined by Eq.\eqref{eq:steady} are not linear, this expression is still not correct.
We need to let the chemical system evolve to reach a valid state. 
This is numerically performed by a few Newton's steps: thanks to the asymptotic formulation, the guess concentration $\vec{C}_\chi(\vec{r},t+\delta t)$ is never too far from such a state, and generally only one or two Newton's steps are sufficient.
  
\section{References}
\begin{thebibliography}{1}

\bibitem[Bouret14]{PLOS}
	Y.Bouret,M.Argentina and L.Counillon,
	\textit{Capturing Intracellular pH Dynamics by Coupling Its Molecular Mechanisms within a Fully Tractable Mathematical Model},
	PLOS One 2014,
	https://doi.org/10.1371/journal.pone.0085449


\bibitem[Bouret16]{BBA}
	L.Counillon, Y.Bouret, I.Marchiq and J.Puyss\'egur,
	\textit{{Na$^+$/H$^+$ antiporter (NHE1) and Lactate/H$^+$ symporters (MCTs)
	 in pH Homeostasis and Cancer Metabolism}},
	 BBA Molecular Cell Research 2016,
	 DOI: 10.1016/j.bbamcr.2016.02.018

\end{thebibliography}

\end{document}
