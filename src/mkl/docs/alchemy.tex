\documentclass[aps,onecolumn,11pt]{revtex4}
\usepackage{graphicx}
\usepackage{amssymb,amsfonts,amsmath,amsthm}
\usepackage{bm}
\usepackage{xfrac}
\usepackage{chemarr}

\usepackage{bookman}


\begin{document}
\maketitle{Alchemy}

\section{Notations}

Let's define a set of $M$ species $A_1,\ldots,A_M$ with a set of $N$ equilibria:
\begin{equation}
	\forall i\in[1:N], \sum_{j=1}^M \nu_{ij}^r A_j \xrightleftharpoons[k^d_i]{k^f_i} \sum_{j=1}^{M} \nu_{ij}^p A_j, 
	\;\; K_i(t) = \dfrac{\prod_{i=j}^M [A_j]^{\nu^p_{ij}}}{\prod_{i=j}^M [A_j]^{\nu^r_{ij}}} = \dfrac{k^f_i}{k^d_i}
\end{equation}

We define:
\begin{equation}
	\Gamma_i(t) = K_i(t) \prod_{j=1}^M [A_j]^{\nu^r_{ij}} - \prod_{j=1}^M [A_j]^{\nu^p_{ij}}
\end{equation}

\begin{itemize}
\item Each equilibrium can be individually solved by its specific extent.

\item A full equilibrium is defined by:
\begin{equation}
	\vec{\Gamma}(\vec{A}) = \vec{0}
\end{equation}

\item A legal transformation is defined by the topology matrix $\bm{\nu} \in \mathcal{M}_{N,M}$ 
and the vector of individual extents $\vec{\xi} \in \mathbb{R}^N$:
\begin{equation}
	\vec{A} + \bm{\nu}^t \vec{\xi}  \geq \vec{0}
\end{equation}

\item
We may also compute the jacobian of $\vec{\Gamma}$:
\begin{equation}
	\vec{\Gamma}(\vec{A}+\delta\vec{A}) = \vec{\Gamma}(\vec{A}) + \bm{\Phi} \delta\vec{A}, \;\; \bm{\Phi} \in \mathcal{M}_{N,M}
\end{equation}

\item
\begin{equation}
	\dfrac{\partial \Gamma_i}{\partial [A_j]} 
	= K_i \delta_{\nu^r_{ij} } \nu^r_{ij} [A_j]^{(\nu^r_{ij}-1)} \prod_{k\not=j}^M [A_k]^{\nu^r_{ik}}
	- \delta_{\nu^p_{ij} } \nu^p_{ij} [A_j]^{(\nu^p_{ij}-1)} \prod_{k\not=j}^M [A_k]^{\nu^p_{ik}}
\end{equation}

\end{itemize}

\section{Relaxation}

\subsection{For one Reaction}
\begin{equation}
\begin{array}{rcl}
\dot{\xi}_i & = &  \displaystyle k^f_i \left[ \prod_{j=1}^M [A_j]^{\nu^r_{ij}}  \right] - k^d_i \left[ \prod_{i=j}^M [A_j]^{\nu^r_{ij}} \right] \\
\\
 & = &  \displaystyle k^f_i \left[ \prod_{j=1}^M ([A_j]_0 - \nu^r_{ij} \xi_i )^{\nu^r_{ij}}  \right] 
 -   k^d_i \left[ \prod_{j=1}^M ([A_j]_0 + \nu^p_{ij} \xi_i )^{\nu^p_{ij}}  \right] \\
 \\
 & \simeq & \displaystyle
- \underbrace{\left(
  k^f_i \left[ \sum_j \left( {\nu^r_{ij}}^2 [A_j]_0^{\nu^r_{ij}-1} \prod_{k\not=j} [A_k]_0^{\nu^r_{ik}}\right) \right]
 +k^d_i \left[ \sum_j \left( {\nu^p_{ij}}^2 [A_j]_0^{\nu^p_{ij}-1} \prod_{k\not=j} [A_k]_0^{\nu^p_{ik}}\right) \right]
  \right)}_{\dfrac{1}{\tau_i}} \xi_i
 \\
\end{array}
\end{equation}

\subsection{Generic}

\begin{equation}
	\dot{\vec{\xi}} = 
	\underbrace{
	\begin{bmatrix}
	k^d_1 &  &\\
	      & \ddots & \\
	      & & k^d_N\\ 
	\end{bmatrix}
	}_{\bm{\Omega}}
	\vec{\Gamma}(\vec{A}_0 + \bm{\nu}^t \vec{\xi})
	\simeq \bm{\Omega} \bm{\Phi}_0 \bm{\nu}^t \vec{\xi}
\end{equation}
Time constants are inverse of the eigenvalues of the matrix.

\section{Balancing}

Let us define:
\begin{equation}
	\mathbb{A}_j = \left\lbrace
	\begin{array}{cl}
	1 & \text{if $A_j$ is active}\\
	0 & \text{if $A_j$ is spectator}\\
	\end{array}
	\right.
\end{equation}

We want to reach the following state, for \textbf{active} species:
\begin{equation}
	\vec{A} + \bm{\nu}^t \vec{\xi} \geq 0
\end{equation}
meaning
\begin{equation}
	\forall j, \mathbb{A}_j\not=0 \Rightarrow \;\; [A_j] + \vec{v}_j \cdot \vec{\xi} \geq 0
\end{equation}
Among the $N_A\leq M$ active species, we make the distinction between 
\begin{itemize}
\item  $N_L$ \textbf{leading} species, where only one coefficient of $\vec{v}_j$ is not zero,
\item  $N_S$ \textbf{seeking} species, where more than two coefficients of $\vec{v}_j$ are not zeros,
\item  with $N_A=N_L+N_S$.
\end{itemize}
We proceed as follow:
\begin{enumerate}
\item For a set of concentrations $\vec{A}$, the leading species allows the $N_L$ conditions to be adjusted.
If we met those leading conditions, then each equilibrium extent $\xi_i$ is associated with a domaine $\Omega_i$,
Otherwise, the set $\vec{A}$ is fundamentally invalid.

\item We obtain a set of partially balanced concentration $\vec{B}$, and we want to find another extent $\vec{\xi}$ that
fulfils the positivity constraints, with the fact that $\vec{\xi}=\vec{0}$ solves all the leading species.

\item Then we must solve $N_S$ subordinate constraints by choosing $\vec{\xi}$ in its domains $\bigotimes_i \Omega_i$.
\end{enumerate}


%
%\begin{equation}
%\left\lbrace
%\begin{array}{rcl}
%	\psi(x)   & = & \dfrac{1}{2} \left[ \max\left( -x, 0 \right) \right]^2\\
%	\psi'(x)  & = & \min(x,0) \\
%	\psi''(x) & = & \mathtt{H}(-x)\\
%\end{array}
%\right.
%\end{equation} 
 

   
\section{Newton$^+$'s Algorithm}

Starting from $\vec{A}_n\geq\vec{0}$, we want to compute:
\begin{equation}
	\vec{\Gamma}(\vec{A}_n+\delta\vec{A}_n) = \vec{0}
\end{equation}
	
\begin{equation}
	\delta\vec{A}_n = \bm{\nu}^t \vec{\xi}_n
\end{equation}
Leading to
\begin{equation}
	\vec{\Gamma}_n + (\bm{\Phi}_n \bm{\nu}^t ) \vec{\xi}_n \simeq \vec{0}
\end{equation}
 
\end{document}
