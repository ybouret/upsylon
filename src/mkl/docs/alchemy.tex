\documentclass[aps,onecolumn,11pt]{revtex4}
\usepackage{graphicx}
\usepackage{amssymb,amsfonts,amsmath,amsthm}
\usepackage{bm}
\usepackage{xfrac}
\usepackage{chemarr}

\usepackage{bookman}


\begin{document}
\maketitle{Alchemy}

\section{Notations}

Let's define a set of $M$ species $A_1,\ldots,A_M$ with a set of $N$ equilibria:
\begin{equation}
	\forall i\in[1:N], \sum_{j=1}^M \nu_{ij}^r A_j \xrightleftharpoons[k^d_i]{k^f_i} \sum_{j=1}^{M} \nu_{ij}^p A_j, 
	\;\; K_i(t) = \dfrac{\prod_{i=j}^M [A_j]^{\nu^p_{ij}}}{\prod_{i=j}^M [A_j]^{\nu^r_{ij}}} = \dfrac{k^f_i}{k^d_i}
\end{equation}

We define:
\begin{equation}
	\Gamma_i(t) = K_i(t) \prod_{j=1}^M [A_j]^{\nu^r_{ij}} - \prod_{j=1}^M [A_j]^{\nu^p_{ij}}
\end{equation}

\begin{itemize}
\item Each equilibrium can be individually solved by its specific extent.

\item A full equilibrium is defined by:
\begin{equation}
	\vec{\Gamma}(\vec{A}) = \vec{0}
\end{equation}

\item A legal transformation is defined by the topology matrix $\bm{\nu} \in \mathcal{M}_{N,M}$ 
and the vector of individual extents $\vec{\xi} \in \mathbb{R}^N$:
\begin{equation}
	\vec{A} + \bm{\nu}^t \vec{\xi}  \geq \vec{0}
\end{equation}

\item
We may also compute the jacobian of $\vec{\Gamma}$:
\begin{equation}
	\vec{\Gamma}(\vec{A}+\delta\vec{A}) = \vec{\Gamma}(\vec{A}) + \bm{\Phi} \delta\vec{A}, \;\; \bm{\Phi} \in \mathcal{M}_{N,M}
\end{equation}

\item
\begin{equation}
	\dfrac{\partial \Gamma_i}{\partial [A_j]} 
	= K_i \delta_{\nu^r_{ij} } \nu^r_{ij} [A_j]^{(\nu^r_{ij}-1)} \prod_{k\not=j}^M [A_k]^{\nu^r_{ik}}
	- \delta_{\nu^p_{ij} } \nu^p_{ij} [A_j]^{(\nu^p_{ij}-1)} \prod_{k\not=j}^M [A_k]^{\nu^p_{ik}}
\end{equation}

\end{itemize}

\section{Relaxation}

\subsection{For one Reaction}
\begin{equation}
\begin{array}{rcl}
\dot{\xi}_i & = &  \displaystyle k^f_i \left[ \prod_{j=1}^M [A_j]^{\nu^r_{ij}}  \right] - k^d_i \left[ \prod_{i=j}^M [A_j]^{\nu^r_{ij}} \right] \\
\\
 & = &  \displaystyle k^f_i \left[ \prod_{j=1}^M ([A_j]_0 - \nu^r_{ij} \xi_i )^{\nu^r_{ij}}  \right] 
 -   k^d_i \left[ \prod_{j=1}^M ([A_j]_0 + \nu^p_{ij} \xi_i )^{\nu^p_{ij}}  \right] \\
 \\
 & \simeq & \displaystyle
- \underbrace{\left(
  k^f_i \left[ \sum_j \left( {\nu^r_{ij}}^2 [A_j]_0^{\nu^r_{ij}-1} \prod_{k\not=j} [A_k]_0^{\nu^r_{ik}}\right) \right]
 +k^d_i \left[ \sum_j \left( {\nu^p_{ij}}^2 [A_j]_0^{\nu^p_{ij}-1} \prod_{k\not=j} [A_k]_0^{\nu^p_{ik}}\right) \right]
  \right)}_{\dfrac{1}{\tau_i}} \xi_i
 \\
\end{array}
\end{equation}

\subsection{Generic}

\begin{equation}
	\dot{\vec{\xi}} = 
	\underbrace{
	\begin{bmatrix}
	k^d_1 &  &\\
	      & \ddots & \\
	      & & k^d_N\\ 
	\end{bmatrix}
	}_{\bm{\Omega}}
	\vec{\Gamma}(\vec{A}_0 + \bm{\nu}^t \vec{\xi})
	\simeq \bm{\Omega} \bm{\Phi}_0 \bm{\nu}^t \vec{\xi}
\end{equation}
Time constants are inverse of the eigenvalues of the matrix.

\section{Balancing}

We want
\begin{equation}
	\vec{A} + \bm{\nu}^t \vec{\xi} \geq 0
\end{equation}
meaning
\begin{equation}
	\forall j, \;\; [A_j] + \vec{v}_j \cdot \vec{\xi} \geq 0
\end{equation}

We define a scalar balance indicator function $\psi(x)$ which is minimised when $x\geq0$.
We define a scalar balance indicator field $\Psi(\vec{A})$ 
which is minimised when all $\psi([A_j])$ are minimised.
Then starting from $\vec{A}_n$:
\begin{equation}
	\Psi(\vec{A}_n + \delta\vec{A}_n) \simeq \Psi(\vec{A}_n) + \partial_{\vec{A}} \Psi \cdot \delta\vec{A}_n
\end{equation}

\begin{equation}
	\Psi(\vec{A}_n + \bm{\nu}^t \vec{\xi}_n) \simeq \Psi(\vec{A}_n) + (\bm{\nu} \partial_{\vec{A}} \Psi) \cdot  \vec{\xi}_n
\end{equation}

If we use a generic function:
\begin{equation}
\left \lbrace
\begin{array}{rcl}
	f(x)      & \geq &0\\
	f'(x>0)   &> 0 \\
\end{array}
\right.
\end{equation}

The we use:
\begin{equation}
\left\lbrace
\begin{array}{rcl}
	\psi( f(x) ) &= &\max(-f(x),0)\\
	\\
	\psi'( f(x) ) & = &
	\left\lbrace
	\begin{array}{rcl}
		x\leq0 & \to & -f'(x)\\
		x>0    & \to & 0\\
	\end{array}
	\right.
\end{array}
\right.
\end{equation}

\begin{equation}
	\Psi = \dfrac{1}{\alpha}\sum_j \mathbb{A}_j \max(0,-([A_j]+\vec{v}_j \vec{\xi})^\alpha)
\end{equation}

\begin{equation}
	-\partial_{\vec{A}} \Psi (\vec{0}) = 
	\begin{bmatrix}
	\vdots \\
	\mathbb{A}_j \times
	\left\lbrace
	\begin{array}{rclcl}
	[A_j] & \leq & 0 & \to & ([A_j])^{\alpha-1} \\
	[A_j] & >    & 0 & \to & 0 \\
	\end{array}
	\right.
	 \\
	\vdots\\
	\end{bmatrix}
\end{equation}
 
 We look for:
 \begin{equation}
  	\vec{\xi} = -u \bm{\nu}\partial_{\vec{A}} \Psi (\vec{0})
 \end{equation}
 
\section{Newton$^+$'s Algorithm}

Starting from $\vec{A}_n\geq\vec{0}$, we want to compute:
\begin{equation}
	\vec{\Gamma}(\vec{A}_n+\delta\vec{A}_n) = \vec{0}
\end{equation}
	
\begin{equation}
	\delta\vec{A}_n = \bm{\nu}^t \vec{\xi}_n
\end{equation}
Leading to
\begin{equation}
	\vec{\Gamma}_n + (\bm{\Phi}_n \bm{\nu}^t ) \vec{\xi}_n \simeq \vec{0}
\end{equation}
 
\end{document}
