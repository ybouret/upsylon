\documentclass[aps,onecolumn,11pt]{revtex4}
\usepackage{graphicx}
\usepackage{amssymb,amsfonts,amsmath,amsthm}
\usepackage{bm}
\usepackage{xfrac}

\usepackage{dejavu}


\begin{document}

\section{sum of integers}
\begin{equation}
	\sum_{i=1}^n i = \dfrac{n(n+1)}{2}
\end{equation}

\begin{equation}
	\sum_{i=1}^n i^2 = \dfrac{n(n+1)(2n+1)}{6}
\end{equation}

\begin{equation}
	\sum_{i=1}^n i^3 = \left(\dfrac{n(n+1)}{2}\right)^2
\end{equation}

\section{symmetric indices}

Let's assume:
\begin{equation}
	1\leq j \leq i \leq n \Rightarrow \dfrac{n(n+1)}{2} \text{ items }.
\end{equation}
For a given $k\in \left[ 1;\dfrac{n(n+1)}{2}\right]$, we look for
\begin{equation}
	1\leq j \leq i \leq n, \;\; k = \frac{i(i-1)}{2}+j
\end{equation}
For $j=0$, we get the previous sum:
\begin{equation}
	k-1=\dfrac{i(i-1)}{2} \Rightarrow i^2-i - 2(k-1) = 0
\end{equation}
\begin{equation}
	i = \dfrac{1+\sqrt{1+8(k-1)}}{2}, j = k-\frac{i(i-1)}{2}
\end{equation}


\end{document}