\documentclass[aps,onecolumn,11pt]{revtex4}
\usepackage{graphicx}
\usepackage{amssymb,amsfonts,amsmath,amsthm}
\usepackage{bm}
\usepackage{xfrac}

\usepackage{dejavu}


\begin{document}

\section{sum of integers}
\begin{equation}
	\sum_{i=1}^n i = \dfrac{n(n+1)}{2}
\end{equation}

\begin{equation}
	\sum_{i=1}^n i^2 = \dfrac{n(n+1)(2n+1)}{6}
\end{equation}

\begin{equation}
	\sum_{i=1}^n i^3 = \left(\dfrac{n(n+1)}{2}\right)^2
\end{equation}

\section{symmetric indices}

Let's assume:
\begin{equation}
	1\leq j \leq i \leq n \Rightarrow \dfrac{n(n+1)}{2} \text{ items }.
\end{equation}
For a given $k\in \left[ 1;\dfrac{n(n+1)}{2}\right]$, we look for
\begin{equation}
	1\leq j \leq i \leq n, \;\; k = \frac{i(i-1)}{2}+j
\end{equation}
For $j=1$, we get $k$ matching $i$:
\begin{equation}
	k=\dfrac{i(i-1)}{2}+1 \Rightarrow i^2-i - 2(k-1) = 0
\end{equation}
\begin{equation}
\left\lbrace
	\begin{array}{rcl}
	i &=& \dfrac{1+\sqrt{1+8(k-1)}}{2}\\
	\\
	j &=& k-\dfrac{i(i-1)}{2}\\
	\end{array}
\right.
\end{equation}
If $k=k_m+1$, $i=i_m+1$, $j=j_m+1$,
\begin{equation}
	\left\lbrace
	\begin{array}{rcl}
	i_m &=& \dfrac{\sqrt{1+8k_m}-1}{2}\\
	\\
	j_m &=& k_m -\dfrac{i_m(i_m+1)}{2}\\
	\end{array}
\right.
\end{equation}



\end{document}