\documentclass[aps,12pt]{revtex4}
\usepackage{graphicx}
\usepackage{amssymb,amsfonts,amsmath,amsthm}
\usepackage{chemarr}
\usepackage{bm}
\usepackage{pslatex}
\usepackage{bookman}



\begin{document}
\title{Arcs}
\maketitle
	

\subsection{Metrics}

\subsubsection{Description}
We have a set of $N$ points $\vec{P}_1,\ldots,\vec{P}_N$, each point in $\mathbb{R}^d$ (where $d$ is the space dimension), and we assume
that they describe a virtual arc for a dimensionless coordinate $t_i=i$.
At each point $\vec{P}_i$ we define a velocity $\vec{V}_i$ and and acceleration $\vec{A}_i$.
\begin{itemize}
\item With a view to compute the curvature, the arc must have a continuous acceleration.
\item To have a consistent metrics, we need two extra degrees of freedom (to match both velocities and position by integration).
\end{itemize}

\subsubsection{Acceleration}

\begin{equation}
	\vec{A}(\tau) = (1-\tau)\vec{A}_i + \tau  \vec{A}_{i+1} + \left[ \tau - \tau^3 \right] \vec{\alpha} + \left[ (1-\tau) - (1-\tau)^3 \right] \vec{\beta}
\end{equation}

\subsubsection{Velocity}
\begin{itemize}
\item Integration:
\begin{equation}
	\vec{V}(t) = \vec{V}_i + \int_0^\tau \vec{A}(u)\mathrm{d}u = \vec{V}_i + \left(\tau-\dfrac{\tau^2}{2}\right) \vec{A}_i + \dfrac{\tau^2}{2} \vec{A}_{i+1} + \left(\dfrac{\tau^2}{2} - \dfrac{\tau^4}{4} \right) \vec{\alpha} + \left(\dfrac{\tau^4}{4} - \tau^3 + \tau^2\right) \vec{\beta}
\end{equation}

\item Expression:
\begin{equation}
\left\lbrace
\begin{array}{rcl}
 \vec{V}(t) & = &\vec{V}_i + \tau 
 \left[ \left(1-\dfrac{\tau}{2}\right) \vec{A}_i + \left(\dfrac{\tau}{2}\right) \vec{A}_{i+1} +
  \tau \left [
  \left(\dfrac{1}{2} - \dfrac{\tau^2}{4} \right) \vec{\alpha} + \left(\dfrac{\tau^2}{4} - \tau + 1\right) \vec{\beta}
  \right ]
 \right]\\
 \\
  & = & \vec{V}_i + \tau 
 \left[ \left(1-\dfrac{\tau}{2}\right) \vec{A}_i + \dfrac{\tau}{2} \vec{A}_{i+1} +
  \tau \left [
  \left(\dfrac{1}{2} -  \left(\dfrac{\tau}{2}\right)^2 \right) \vec{\alpha} + \left( 1-\tau \left(1-\dfrac{\tau}{4}\right)\right) \vec{\beta}
  \right ]
 \right]\\
 \end{array}
 \right.
\end{equation}

\item Equation:
\begin{equation}
\vec{V}_{i+1} = \vec{V}_i + \dfrac{1}{2}\left(\vec{A}_i + \vec{A}_{i+1}\right) + \dfrac{1}{4} \left( \vec{\alpha} + \vec{\beta}\right)
\end{equation}

\end{itemize}

\subsubsection{Position}
\begin{itemize}
\item Integration:
\begin{equation}
	\vec{P}(t) = \vec{P}_{i} + \int_0^\tau \vec{V}(u)\mathrm{d}u = \vec{P}_i + \tau \vec{V}_i 
	+ \left(\dfrac{\tau^2}{2} - \dfrac{\tau^3}{6}\right) \vec{A}_i + \dfrac{\tau^3}{6} \vec{A}_{i+1} 
	+ \left(\dfrac{\tau^3}{6} - \dfrac{\tau^5}{20} \right) \vec{\alpha}
	+ \left( \dfrac{\tau^5}{20} - \dfrac{\tau^4}{4} + \dfrac{\tau^3}{3} \right) \vec{\beta}
\end{equation}

\item Equation:
\begin{equation}
	\vec{P}_{i+1} = \vec{P}_i + \vec{V}_{i} + \dfrac{1}{3} \vec{A}_i + \dfrac{1}{6} \vec{A}_{i+1} + \dfrac{7}{60} \vec{\alpha} + \dfrac{2}{15} \vec{\beta}
\end{equation}

\end{itemize}

\subsubsection{Coefficients}

\begin{equation}
\begin{pmatrix}
\dfrac{1}{4} & \dfrac{1}{4} \\
\\
\dfrac{7}{60} & \dfrac{2}{15}\\
\end{pmatrix}
\begin{pmatrix}
\vec{\alpha}\\
\vec{\beta}\\
\end{pmatrix}
=
\begin{pmatrix}
\left[\vec{V}_{i+1}-\vec{V}_{i}\right] - \dfrac{1}{2}\left[\vec{A}_i + \vec{A}_{i+1}\right]\\
\left[\vec{P}_{i+1}-\vec{P}_{i}\right] - \vec{V}_i - \dfrac{1}{3} \vec{A}_i - \dfrac{1}{6} \vec{A}_{i+1}\\
\end{pmatrix}
\end{equation}

\begin{equation}
\begin{pmatrix}
\vec{\alpha}\\
\vec{\beta}\\
\end{pmatrix}
=
\begin{pmatrix}
32  & -60 \\
-28 & 60\\
\end{pmatrix}
\begin{pmatrix}
\left[\vec{V}_{i+1}-\vec{V}_{i}\right] - \dfrac{1}{2}\left[\vec{A}_i + \vec{A}_{i+1}\right]\\
\left[\vec{P}_{i+1}-\vec{P}_{i}\right] - \vec{V}_i - \dfrac{1}{3} \vec{A}_i - \dfrac{1}{6} \vec{A}_{i+1}\\
\end{pmatrix}
\end{equation}

\begin{equation}
\begin{array}{rcrrr}
	\vec{\alpha} & = & -60\left[\vec{P}_{i+1}-\vec{P}_i\right] & + \left[32\vec{V}_{i+1} + 28 \vec{V}_i\right] &  - \left[6\vec{A}_{i+1} -4\vec{A}_{i}\right]\\
	\vec{\beta}  & = & 60\left[\vec{P}_{i+1}-\vec{P}_i\right]   & - \left[28\vec{V}_{i+1} + 32 \vec{V}_i\right] & + \left[4\vec{A}_{i+1}-6\vec{A}_{i}\right]\\
\end{array}
\end{equation}

\subsection{Kinematics}

\subsubsection{Bulk}

\begin{equation}
\forall 1 < i < N, \;\;
\left\lbrace
\begin{array}{rcll}
	\vec{V}_i & = & \dfrac{1}{2} \left[ \vec{P}_{i+1} - \vec{P}_i\right] & \\
	\\
	\vec{A}_i & = &  \left[ \vec{P}_{i+1} - 2\vec{P}_i + \vec{P}_{i+1} \right] & \text{or } \dfrac{1}{2}\left[\vec{V}_{i+1}-\vec{V}_{i-1}\right]
\end{array}
\right.
\end{equation}

\subsubsection{Standard Boundaries for $N=2$}
\begin{equation}
\left\lbrace
\begin{array}{rcll}
\vec{V}_1 & = & \text{given, or} & \vec{P}_2 - \vec{P}_1\\ 
\vec{V}_2 & = & \text{given, or} & \vec{P}_2 - \vec{P}_1 = \vec{V}_1\\
\vec{A}_1 & = & \text{given, or} & \vec{V}_2 - \vec{V}_1\\ 
\vec{A}_N & = & \text{given, or} & \vec{V}_2 - \vec{V}_1=\vec{A}_1\\ 
\end{array} 
\right.
\end{equation}

\subsubsection{Standard Boundaries for $N>2$}
\begin{equation}
\left\lbrace
\begin{array}{rcll}
\vec{V}_1 & = & \text{given, or} & \dfrac{1}{2}\left( \left[\vec{P}_2 - \vec{P}_3\right] + 3\left[\vec{P_2}-\vec{P}_1\right] \right)\\ 
\\
\vec{V}_N & = & \text{given, or} & \dfrac{1}{2}\left( \left[\vec{P}_{N-2}-\vec{P}_{N-1}\right]  + 3\left[\vec{P}_{N} - \vec{P}_{N-1}\right] \right)\\
\\
\vec{A}_1 & = & \text{given, or} &  \vec{A}_2=\left[\vec{P}_1 - 2\vec{P}_2 + \vec{P}_3\right] \text{, or } \dfrac{1}{2}\left( \left[\vec{V}_2 - \vec{V}_3\right] + 3\left[\vec{V_2}-\vec{V}_1\right] \right)\\ 
\\
\vec{A}_N & = & \text{given, or} &   \vec{A}_{N-1}=\left[\vec{P}_{N-2} - 2\vec{P}_{N-1} + \vec{P}_{N}\right] \text{, or } \dfrac{1}{2}\left( \left[\vec{V}_{N-2}-\vec{V}_{N-1}\right]  + 3\left[\vec{V}_{N} - \vec{V}_{N-1}\right] \right)\\ 
\end{array}
\right.
\end{equation}


\subsubsection{Periodic Boundaries for $N=2$}
\begin{equation}
\left\lbrace
\begin{array}{rcll}
\vec{V}_1 & = & \text{given, or} & \dfrac{1}{2}\left(\vec{P}_2 - \vec{P}_{N-1}\right)\\ 
\\
\vec{V}_N & = & \text{given, or} & \dfrac{1}{2}\left(\vec{P}_1 - \vec{P}_{N-1}\right)\\
\\
\vec{A}_1 & = & \text{given, or} &   \left[\vec{P}_{N} - 2\vec{P}_{1} + \vec{P}_2\right]  \text{, or } \dfrac{1}{2}\left(\vec{V}_2 - \vec{V}_{N-1}\right)\\ 
\\
\vec{A}_N & = & \text{given, or} &    \left[\vec{P}_{N-1} - 2\vec{P}_N + \vec{P}_1\right] \text{, or }  \dfrac{1}{2}\left(\vec{V}_1 - \vec{V}_{N-1}\right)\\ 
\end{array} 
\right.
\end{equation}

\begin{equation}
\vec{P} = (1-\tau) \vec{P}_{i} + \tau \vec{P}_{i+1} + \sum_{i=1}^4 \tau^i\vec{a}_i 
\end{equation}

\begin{equation}
\begin{array}{rcl}
\vec{V} & = & \vec{P}_{i+1} - \vec{P}_i + \sum_{i=1}^4 i\tau^{i-1}\vec{a}_i \\
& = & (1-\tau) \vec{V}_{i+1} + \tau \vec{V}_{i}\\
\end{array}
\end{equation}

\end{document}
