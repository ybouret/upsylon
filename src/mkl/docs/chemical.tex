\documentclass[aps,onecolumn,11pt]{revtex4}
\usepackage{graphicx}
\usepackage{amssymb,amsfonts,amsmath,amsthm}
\usepackage{bm}
\usepackage{xfrac}
\usepackage{chemarr}

\usepackage{bookman}


\begin{document}
\title{Alchemy}
\maketitle

\section{Notations}

Let's define a set of $M$ species $A_1,\ldots,A_M$ with a set of $N$ equilibria:
\begin{equation}
	\forall i\in[1:N], \sum_{j=1}^M \nu_{ij}^r A_j \xrightleftharpoons[k^d_i]{k^f_i} \sum_{j=1}^{M} \nu_{ij}^p A_j, 
	\;\; K_i(t) = \dfrac{\prod_{i=j}^M [A_j]^{\nu^p_{ij}}}{\prod_{i=j}^M [A_j]^{\nu^r_{ij}}} = \dfrac{k^f_i}{k^d_i}
\end{equation}

We define:
\begin{equation}
	\Gamma_i(t) = K_i(t) \prod_{j=1}^M [A_j]^{\nu^r_{ij}} - \prod_{j=1}^M [A_j]^{\nu^p_{ij}}
\end{equation}

\begin{itemize}
\item Each equilibrium can be individually solved by its specific extent.

\item A full equilibrium is defined by:
\begin{equation}
	\vec{\Gamma}(\vec{A}) = \vec{0}
\end{equation}

\item A legal transformation is defined by the topology matrix $\bm{\nu} \in \mathcal{M}_{N,M}$ 
and the vector of individual extents $\vec{\xi} \in \mathbb{R}^N$:
\begin{equation}
	\vec{A} + \bm{\nu}^t \vec{\xi}  \geq \vec{0}
\end{equation}

\item
We may also compute the jacobian of $\vec{\Gamma}$:
\begin{equation}
	\vec{\Gamma}(\vec{A}+\delta\vec{A}) = \vec{\Gamma}(\vec{A}) + \bm{\Phi} \delta\vec{A}, \;\; \bm{\Phi} \in \mathcal{M}_{N,M}
\end{equation}

\item
\begin{equation}
	\dfrac{\partial \Gamma_i}{\partial [A_j]} 
	= K_i \delta_{\nu^r_{ij} } \nu^r_{ij} [A_j]^{(\nu^r_{ij}-1)} \prod_{k\not=j}^M [A_k]^{\nu^r_{ik}}
	- \delta_{\nu^p_{ij} } \nu^p_{ij} [A_j]^{(\nu^p_{ij}-1)} \prod_{k\not=j}^M [A_k]^{\nu^p_{ik}}
\end{equation}

\end{itemize}

\section{Relaxation}

\subsection{For one Reaction}
\begin{equation}
\begin{array}{rcl}
\dot{\xi}_i & = &  \displaystyle k^f_i \left[ \prod_{j=1}^M [A_j]^{\nu^r_{ij}}  \right] - k^d_i \left[ \prod_{i=j}^M [A_j]^{\nu^r_{ij}} \right] \\
\\
 & = &  \displaystyle k^f_i \left[ \prod_{j=1}^M ([A_j]_0 - \nu^r_{ij} \xi_i )^{\nu^r_{ij}}  \right] 
 -   k^d_i \left[ \prod_{j=1}^M ([A_j]_0 + \nu^p_{ij} \xi_i )^{\nu^p_{ij}}  \right] \\
 \\
 & \simeq & \displaystyle
- \underbrace{\left(
  k^f_i \left[ \sum_j \left( {\nu^r_{ij}}^2 [A_j]_0^{\nu^r_{ij}-1} \prod_{k\not=j} [A_k]_0^{\nu^r_{ik}}\right) \right]
 +k^d_i \left[ \sum_j \left( {\nu^p_{ij}}^2 [A_j]_0^{\nu^p_{ij}-1} \prod_{k\not=j} [A_k]_0^{\nu^p_{ik}}\right) \right]
  \right)}_{\dfrac{1}{\tau_i}} \xi_i
 \\
\end{array}
\end{equation}

\subsection{Generic}

\begin{equation}
	\dot{\vec{\xi}} = 
	\underbrace{
	\begin{bmatrix}
	k^d_1 &  &\\
	      & \ddots & \\
	      & & k^d_N\\ 
	\end{bmatrix}
	}_{\bm{k}_d}
	\vec{\Gamma}(\vec{A}_0 + \bm{\nu}^t \vec{\xi})
	\simeq \bm{k}_d \bm{\Phi}_0 \bm{\nu}^t \vec{\xi}
\end{equation}
Time constants are inverse of the eigenvalues of the matrix before $\vec{\xi}$.

\section{Balancing}

 \begin{itemize}
\item All species that are used only once in the system are \textbf{primary} species which offers single limiting extent for the equilibrium in which they
are involved.\\
Solving such a constraint is achieved by carefully moving (a.k.a taking care of vanishing species of) each equilibrium corresponding to the primary constraint.

\item All species that area used more than once in the system are \textbf{replica} species.
\end{itemize}

   
\end{document}
