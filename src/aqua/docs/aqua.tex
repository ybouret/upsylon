\documentclass[aps,12pt]{revtex4}
\usepackage{graphicx}
\usepackage{amssymb,amsfonts,amsmath,amsthm}
\usepackage{chemarr}
\usepackage{bm}
\usepackage{pslatex}
\usepackage{mathptmx}
\usepackage{xfrac}
\usepackage{bookman}

%%  
\newcommand{\trn}[1]{{#1}^{\mathtt{T}}}
\newcommand{\conc}[1]{{\left[#1\right]}}
 
\begin{document}
\section{Species and Equilibria}
Let us consider an aqueous system with $M$ species $A_1,\ldots,A_M$.
Those $M$ species are involved in $N$ equilibria:
\begin{equation}
	\forall j\in[1:N], \; \sum_{i=1}^M \nu_{j,i} \conc{A_i} = 0,
	 \;\; K_j = \prod_{i=1}^M \conc{A_i}^{\nu_{ji}}
\end{equation}

\section{Chemical Extent}
We consider the chemical topology matrix:
\begin{equation}
	\bm{\nu} \in \mathcal{M}
\end{equation}

\end{document}