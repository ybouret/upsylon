\documentclass[aps]{revtex4}
\usepackage{graphicx}
\usepackage{amssymb,amsfonts,amsmath,amsthm}
\usepackage{chemarr}
\usepackage{bm}
\usepackage{pslatex}
\usepackage{mathptmx}
\usepackage{xfrac}

%% concentration notations
\newcommand{\mymat}[1]{\boldsymbol{#1}}
\newcommand{\mytrn}[1]{~^{\mathsf{t}}\!{#1}}
\newcommand{\myvec}[1]{\overrightarrow{#1}}
\newcommand{\mygrad}{\vec{\nabla}}
\newcommand{\myhess}{\mathcal{H}}


\begin{document}
\title{Chemical System}
\maketitle

\section{Notation}
We assume that we have $M$ species $(A_1,\ldots,A_M)$ coupled
by $N$ equilibria such that
\begin{equation}
	0 = \sum_j \mymat{\nu_{i,j}} A_j,\;\; K_i = \prod_j C_j^{\mymat{\nu}_{i,j}}.
\end{equation}
An equilibria is met when
\begin{equation}
	\Gamma_i = \left(K_i \prod_{\nu_{i,j}<0} C_j^{-\nu_{i,j}}\right) - \left(\prod_{\nu_{i,j}>0} C_j^{\nu_{i,j}}\right)
\end{equation}
equals $0$.\\
We also define the gradient
\begin{equation}
	\vec{\Phi}_i = \partial_{\vec{C}}{\Gamma_i}
\end{equation}

\section{Properties of one equilibrium}
For a given set of concentrations $\vec{C}$, the composition may evolve according to the chemical extent $\xi_i$
such that
\begin{equation}
	\vec{C} = \vec{C}_0 + \xi_i \cdot \vec{\nu}_i
\end{equation}
so that
\begin{equation}
	\Gamma_i\left(\vec{C} = \vec{C}_0 + \xi_i  \vec{\nu}_i\right) 
	\approx
	\Gamma_i\left(\vec{C}_0\right) + \xi_i \underbrace{\left(\vec{\Phi}_i\vec{\nu}_i\right)}_{\leq 0}
\end{equation}

\end{document}