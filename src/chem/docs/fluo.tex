\documentclass[aps]{revtex4}
\usepackage{graphicx}
\usepackage{amssymb,amsfonts,amsmath,amsthm}
\usepackage{chemarr}
\usepackage{bm}
\usepackage{pslatex}
\usepackage{mathptmx}
\usepackage{xfrac}

\newcommand{\mychem}[1]{\mathtt{#1}}
\newcommand{\myconc}[1]{\left\lbrack{#1}\right\rbrack}
\newcommand{\plus}{\mychem{+}}
\newcommand{\proton}{\mychem{H}^\plus}

\begin{document}

\title{Spectro}
\maketitle

\section{Chemistry}

We have a probe
\begin{equation}
	\alpha \xrightleftharpoons{} \beta + \proton, \;\; K = \dfrac{\myconc{\beta}\myconc{\proton}}{ \myconc{\alpha} }
\end{equation}
the intensity emitted from a volume $V$ is expressed as
\begin{equation}
	I = V \left( \Phi_\alpha \myconc{\alpha}  + \Phi_\beta \myconc{\beta} \right)
\end{equation}
For a local concentration $C_0$ in probe and a given $h=\myconc{\proton}$,
\begin{equation}
	 \myconc{\alpha}  = C_0 \dfrac{h}{K+h},\; \myconc{\beta} = C_0 \dfrac{K}{K+h}
\end{equation}

\section{Calibration}
We take two calibration pHs as far as possible from the measuring zone, let's say $h=a$ and $h=b$, so that
\begin{equation}
\left\lbrace
\displaystyle
\begin{array}{rcl}
	I_a & = & V C_0 \left[  \Phi_\alpha \dfrac{a}{K+a} + \Phi_\beta \dfrac{K}{K+a}\right]\\
	\\
	I_b & = & V C_0 \left[  \Phi_\alpha \dfrac{b}{K+b} + \Phi_\beta \dfrac{K}{K+b}\right]\\
\end{array}
\right.
\end{equation}
and using $J_x=I_x/(VC_0)$ we are left with
\begin{equation}
\left\lbrace
\displaystyle
	\begin{array}{rcl}
	a \Phi_\alpha + K \Phi_\beta & = & J_a(K+a)\\
	\\
	b \Phi_\alpha + K \Phi_\beta & = & J_b(K+b)\\
	\end{array}
\right.
\end{equation}
Leading to
\begin{equation}
\left\lbrace
\displaystyle
	\begin{array}{rcl}
	\Phi_\alpha & = & \dfrac{J_a(K+a)-J_b(K+b)}{a-b}\\
	\\
	\Phi_\beta & = & \dfrac{aJ_b(K+b)-bJ_a(K+a)}{K(a-b)}\\
	\end{array}
\right.
\end{equation}

\section{Retrieving pH from Intensity}
We get
\begin{equation}
	J = \dfrac{1}{K+h} \left[ h \Phi_\alpha + K \Phi_\beta \right]
\end{equation}
so that
\begin{equation}
	h = K \left( \dfrac{\Phi_\beta-J}{J-\Phi_\alpha} \right)
\end{equation}
and
\begin{equation}
	pH = pK + \log \left( \dfrac{\Phi'_\alpha-I}{I-\Phi'_\beta}\right)
\end{equation}
with
\begin{equation}
	\left\lbrace
\displaystyle
	\begin{array}{rcl}
	\Phi'_\alpha & = & \dfrac{I_a(K+a)-I_b(K+b)}{a-b}\\
	\\
	\Phi'_\beta & = & \dfrac{aI_b(K+b)-bI_a(K+a)}{K(a-b)}\\
	\end{array}
\right.
\end{equation}

\end{document}
