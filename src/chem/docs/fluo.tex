\documentclass[aps]{revtex4}
\usepackage{graphicx}
\usepackage{amssymb,amsfonts,amsmath,amsthm}
\usepackage{chemarr}
\usepackage{bm}
\usepackage{pslatex}
\usepackage{mathptmx}
\usepackage{xfrac}

\newcommand{\mychem}[1]{\mathtt{#1}}
\newcommand{\myconc}[1]{\left\lbrack{#1}\right\rbrack}
\newcommand{\plus}{\mychem{+}}
\newcommand{\proton}{\mychem{H}^\plus}

\begin{document}

\title{Fluo}
\maketitle

\section{Chemistry}
We have a probe
\begin{equation}
	\alpha \xrightleftharpoons{} \beta + \proton, \;\; K = \dfrac{\myconc{\beta}\myconc{\proton}}{ \myconc{\alpha} }
\end{equation}
with a total (local) concentration 
\begin{equation}
C_0 = \myconc{\alpha} + \myconc{\beta}.
\end{equation}
The intensity emitted from a volume $V$ for a wavelength $\lambda$ is expressed as
\begin{equation}
	I(\lambda) = V \left( \phi_\alpha(\lambda) \myconc{\alpha}  + \phi_\beta(\lambda) \myconc{\beta} \right).
\end{equation}
The isobestic wavelength $\lambda_{iso}$  is such that 
\begin{equation}
	\phi_\alpha(\lambda_{iso}) \simeq \phi_\beta(\lambda_{iso}) \simeq \phi_{iso}
\end{equation}
and
\begin{equation}
	I_{iso} = V  \phi_{iso} C_0
\end{equation}
is independent of the pH value.

For a local concentration $C_0$ in probe and a given $h=\myconc{\proton}$ we get
\begin{equation}
\left\lbrace
\begin{array}{rcl}
	 \myconc{\alpha}  & = & C_0 \dfrac{h}{K+h}\\
	 \\
	 \myconc{\beta}   & = & C_0 \dfrac{K}{K+h}\\
\end{array}
\right.
\end{equation}
hence defining
\begin{equation}
	F = \dfrac{I(\lambda)}{I_{iso}} = \Phi_\alpha \dfrac{h}{K+h} + \Phi_\beta \dfrac{K}{K+h}
\end{equation}
with $\Phi_x = \phi_x/\phi_{iso}$.
\section{Calibration}
\noindent We need to estimate $\Phi_\alpha$ and $\Phi_\beta$, so we need two independent values of $F$.\\
We take two calibration pHs as far as possible from the measuring zone, let's say $h=a$ (for acid) and $h=b$ (for base), so that
\begin{equation}
\left\lbrace
\displaystyle
\begin{array}{rcl}
	F_a & = &  \left[  \Phi_\alpha \dfrac{a}{K+a} + \Phi_\beta \dfrac{K}{K+a}\right]\\
	\\
	F_b & = & \left[  \Phi_\alpha \dfrac{b}{K+b} + \Phi_\beta \dfrac{K}{K+b}\right]\\
\end{array}
\right.
\end{equation}
We note that 
\begin{equation}
	\min\left(\Phi_\alpha,\Phi_\beta\right) < F  < \max\left(\Phi_\alpha,\Phi_\beta\right)
\end{equation}
since $\Phi_\alpha$ is the most acid ratio and $\Phi_\beta$ is the most basic ratio.

We need to solve
\begin{equation}
\left\lbrace
\displaystyle
	\begin{array}{rcl}
	a \Phi_\alpha + K \Phi_\beta & = & F_a(K+a)\\
	\\
	b \Phi_\alpha + K \Phi_\beta & = & F_b(K+b)\\
	\end{array}
\right.
\end{equation}
leading to
\begin{equation}
\left\lbrace
\displaystyle
	\begin{array}{rcl}
	\Phi_\alpha & = & \dfrac{F_a(K+a)-F_b(K+b)}{a-b}\\
	\\
	\Phi_\beta & = & \dfrac{aF_b(K+b)-bF_a(K+a)}{K(a-b)}\\
	\end{array}
\right.
\end{equation}

\section{Retrieving pH from Intensity}
We get
\begin{equation}
	F = \dfrac{1}{K+h} \left[ h \Phi_\alpha + K \Phi_\beta \right]
\end{equation}
so that
\begin{equation}
	h = K \left( \dfrac{\Phi_\beta-F}{F-\Phi_\alpha} \right)
\end{equation}
and
\begin{equation}
	pH = pK + \log \left( \dfrac{\Phi_\alpha-F}{F-\Phi_\beta}\right) = pK + \log \left( \dfrac{F-\Phi_\alpha}{\Phi_\beta-F}\right)
\end{equation}
\begin{itemize}
\item Since, theoretically, there cannot be any value of $F$ exceeding $\Phi_\alpha$ or $\Phi_\beta$, this formula holds for the whole set or experimental ratios $F$.
\item But to have the most precise values of $\Phi_{\alpha,\beta}$, one need to have the greatest possible $a-b$. 
\item For exemple, if $\Phi_\alpha>\Phi_\beta$ then $F_a>F_b$, leading to $\Phi_\alpha>F_a$ and $\Phi_\beta<F_b$, and vice versa if $\Phi_\alpha<\Phi_\beta$. 
So, from any experimental data within the whole calibration, the pH is always defined !
\end{itemize}




\end{document}
