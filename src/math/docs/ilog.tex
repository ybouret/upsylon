\documentclass[aps]{revtex4}
\usepackage{graphicx}
\usepackage{amssymb,amsfonts,amsmath,amsthm}
\usepackage{chemarr}
\usepackage{bm}
\usepackage{pslatex}
\usepackage{mathptmx}
\usepackage{xfrac}

%% concentration notations
\newcommand{\mymat}[1]{\boldsymbol{#1}}
\newcommand{\mytrn}[1]{~^{\mathsf{t}}\!{#1}}
\newcommand{\myvec}[1]{\overrightarrow{#1}}
\newcommand{\mygrad}{\vec{\nabla}}
\newcommand{\myhess}{\mathcal{H}}

\newcommand{\myfloor}[1]{{\left\lfloor#1\right\rfloor}}
\newcommand{\myceil}[1]{{\left\lceil#1\right\rceil}}


\begin{document}
\title{Integer Logarithm approximation}
\maketitle

\section{Dyadic Majoration}
We evaluate $\log(x)$ on the set of $x_i=2^i$.
Since the function is slowly varying, we use self similar cubic approximation:
\begin{equation}
\left\lbrace
\begin{array}{rcl}
	L_i(x)  & = & \ln(x_i) + \left(\dfrac{x-x_i}{x_i}\right) - b \left(\dfrac{x-x_i}{x_i}\right)^2 + c\left(\dfrac{x-x_i}{x_i}\right)^3\\
	\\
	L_i'(x) & = & \dfrac{1}{x_i} - 2\dfrac{b}{x_i} \left(\dfrac{x-x_i}{x_i}\right) + 3 \dfrac{c}{x_i^2}  \left(\dfrac{x-x_i}{x_i}\right)^2 \\
\end{array}
\right.
\end{equation}
leading to:
\begin{equation}
\left\lbrace
\begin{array}{rcl}
	\ln(x_{i+1})       & = & L(x_{i_1})\\
	\dfrac{1}{x_{i+1}} & = & L'(x_{i+1}) \\
\end{array}
\right.
\end{equation}

\begin{equation}
\left\lbrace
\begin{array}{rcl}
b & = & \dfrac{5}{2}-3\ln(2)\\
\\
c & = & \dfrac{3}{2}-2\ln(2)\\
\end{array}
\right.
\end{equation}
and
\begin{equation}
\boxed{
\forall x\in[2^i,2^{i+1}], \;\; \ln(x) \leq L_i(x) 
}
\end{equation}


\begin{equation}
	L_i(x) = \dfrac{
	2^{3i}\ln(x_i) 
	+ 2^{2i}\left(x-2^i\right)
	- b 2^{i}\left(x-2^i\right)^2
	+ c      \left(x-2^i\right)^3
	}
	{2^{3i}} 
\end{equation}
We find that the maximum of $L_i(x)-\ln(x)$ is for 
\begin{equation}
	\dfrac{x_{i,max}-2^i}{2^i} = \dfrac{12\ln(2)-8}{9-12\ln(2)}
\end{equation}
and the value is independent of $i$
\begin{equation}
	\forall i, \;\; \max\lbrace L_i(x)-\ln(x)\rbrace \simeq 0.003642013011295071
\end{equation}

\end{document}

Let us use a binary level $2^p$
\begin{equation}
	L_i(x) \leq \Lambda_i(x,p) = 
	\dfrac{ i \cdot 2^{3i} \myceil{2^p \ln(2)} 
	+ 2^{2i+p} \left(x-2^i\right)
	- \myfloor{2^p b} 2^{i}\left(x-2^i\right)^2 + \myceil{2^p c} \left(x-2^i\right)^3
	}
	{2^{3i+p}}
\end{equation}
We compute
\begin{equation}
\left\lbrace
\begin{array}{rcl}
	\Lambda_i(2^{i+1},p) & = & \left( i\myceil{2^p \ln(2)} + 2^p - \myfloor{2^p b} + \myceil{2^p c} \right) 2^{-p}\\
	\Lambda_{i+1}(2^{i+1},p) & = & \left( (i+1) \myceil{2^p \ln(2)} \right) 2^{-p}\\
\end{array}
\right.
\end{equation}
To be coherent
\begin{equation}
	\left\lbrace
\begin{array}{rl}
	 & \Lambda_i(2^{i+1},p)  \leq  \Lambda_{i+1}(2^{i+1},p)\\
\Leftrightarrow & 2^p + \myceil{2^p c} - (\myceil{2^p \ln(2)} + \myfloor{2^p b}) \leq 0 \\
\end{array}
\right.
\end{equation}
and we find a reduced set of solutions, overriding multiple values:
\begin{equation}
	p \in \lbrace 11,13,14,16,17 \rbrace
\end{equation}

\begin{equation}
\begin{array}{cccc}
p & \myceil{2^p \ln(2)} &  \myfloor{2^p b} & \myceil{2^p c}\\
11 & 1420 & 861 & 233\\
13 & 5679 & 3445 & 932\\
14 & 11357 & 6890 & 1863\\
16 & 45427 & 27561 & 7452\\
17 & 90853 & 55123 & 14904\\
\end{array}
\end{equation}

\section{Dyadic Minoration}
Using the convexity
\begin{equation}
 	\ln(x) \geq C_i(x) = \ln(x_i) + \dfrac{x-x_i}{x_{i+1}-{x_i}} \left[ \ln(x_{i+1}) - \ln(x_i)\right] + A_i (x_{i+1}-x)(x-x_{i})
\end{equation}
We want
\begin{equation}
D_i(x)=\ln(x)-C_i(x)\geq0
\end{equation}
The two boundaries conditions are
\begin{equation}
D_i'(x_i)\geq 0 \text{ and } D_i'(x_{i+1})\leq0
\end{equation}
\begin{equation}
	D_i'(x) = \dfrac{1}{x} - \left[ \underbrace{\dfrac{\ln(x_{i+1}) - \ln(x_i)}{x_{i+1}-x_i}}_{\sigma_i} + A_i \left(x_{i+1}+x_i-2x\right)\right]
\end{equation}
\begin{equation}
\left\lbrace
\begin{array}{rcl}
\dfrac{1}{x_i}     - \left[\sigma_i + A_i\left(x_{i+1}-x_i\right)\right] & \geq & 0 \\
\dfrac{1}{x_{i+1}} - \left[\sigma_i - A_i\left(x_{i+1}-x_i\right)\right] & \leq & 0 \\
\end{array}
\right.
\end{equation}
which provides
\begin{equation}
	A_i = \dfrac{\ln(2)-\frac{1}{2}}{2^{2i}}
\end{equation}

\end{document}


