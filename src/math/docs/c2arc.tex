\documentclass[aps,12pt]{revtex4}
\usepackage{graphicx}
\usepackage{amssymb,amsfonts,amsmath,amsthm}
\usepackage{chemarr}
\usepackage{bm}
\usepackage{pslatex}
\usepackage{bookman}



\begin{document}
\title{Minimal $\mathcal{C}^2$ arcs}
\maketitle

\section{Definitions}
We have a set of $N$ points $\vec{P}_1,\ldots,\vec{P}_N$, and we assume
that we are able to compute the respective velocities  $\vec{V}_1,\ldots,\vec{V}_N$.\\
To produce a $\mathcal{C}^2$ arc from these data, 
me must assign a field of accelerations $\vec{A}_1,\ldots,\vec{A}_N$. ,which imposes and extraneous degree of freedom for each segment.

\section{Accelerations}
On a segment $\vec{P}_i \to \vec{P}_{i+1}$ we get:
\begin{equation}
	\vec{A}(t) = \vec{A}_{i} \left(1-t\right) + t \vec{A}_{i+1} + t\left(1-t\right) \vec{Q}_i
\end{equation}

\section{Velocities}
On a segment $\vec{P}_i \to \vec{P}_{i+1}$ we get:
\begin{equation}
	\vec{V}(t) = \vec{V}_{i} + \int_0^t \vec{A}(u)\,\mathrm{d}u = 
	\vec{V}_{i} + 
	\left(t-\frac{t^2}{2}\right) \vec{A}_{i} 
	+ \frac{t^2}{2} \vec{A}_{i+1}
	+ \left(\frac{t^2}{2} - \frac{t^3}{3}\right) \vec{Q}_i,
\end{equation}
so that:
\begin{equation}
	\vec{V}_{i+1} = \vec{V}_{i} + \frac{1}{2} \left(\vec{A}_i+\vec{A}_{i+1}\right) + \frac{1}{6} \vec{Q}_i.
\end{equation}
Finally, the quadratic term is:
\begin{equation}
\boxed{
	\vec{Q}_i = 6\left(\vec{V}_{i+1}-\vec{V}_{i}\right) - 3  \left(\vec{A}_i+\vec{A}_{i+1}\right) 
}
\end{equation}

\section{Position}
We obtain:
\begin{equation}
	\vec{P}(t) = \vec{P}_{i} + \int_0^t \vec{V}(u)  \,\mathrm{d}u = 
	\vec{P}_i + t \vec{V}_i + \left(\frac{t^2}{2}-\frac{t^3}{6}\right) \vec{A}_{i} 
	+ \frac{t^3}{6} \vec{A}_{i+1}
	+ \left(\frac{t^3}{6} - \frac{t^4}{12}\right) \vec{Q}_i.
\end{equation}
The continuity writes:
\begin{equation}
	\vec{P}_{i+1} = \vec{P}_{i} + \vec{V}_i + \frac{1}{3} \vec{A}_i + \frac{1}{6} \vec{A}_{i+1} + \frac{1}{12} \vec{Q}_i
\end{equation}
then:
\begin{equation}
\label{eq:a}
\boxed{
	\vec{P}_{i+1} - \vec{P}_{i} = \frac{1}{2} \left( \vec{V}_i + \vec{V}_{i+1} \right)
	+\frac{1}{12}\left(\vec{A}_{i}-\vec{A}_{i+1}\right)
	}
\end{equation}
This means that  any set of accelerations matching Eq\eqref{eq:a} shall provide a piecewise $\mathcal{C}^2$ arc.
\section{Expressions}
We want to minimize:
\begin{equation}
	\mathcal{E} = \sum_{j=1}^{N} \frac{1}{2} \vec{A}_j^2,
\end{equation}
under the $M$ constraints:
\begin{equation}
	\forall i \in [1:M], \; \vec{0} = \left(\vec{A}_{i}-\vec{A}_{i+1}\right) + 6 \left( \vec{V}_i + \vec{V}_{i+1} \right ) - \left( \vec{P}_{i+1} - \vec{P}_{i} \right) = \vec{S}_i
\end{equation}
We hence minimize:
\begin{equation}
	\mathcal{L} = \mathcal{E} - \sum_{i=1}^M \vec{\lambda}_i \vec{S}_i.
\end{equation}
We get the expression:
\begin{equation}
	\forall j \in [1:N],\;\; \partial_{\vec{A}_j} \mathcal{L} = \vec{A}_j 
	- \sum_{i=1}^M \left( \delta_{i,j} - \delta_{i,j+1}\right)\vec{\lambda}_i
\end{equation}
so that
\begin{equation}
	\vec{A}_j =  \sum_{i=1}^M \left( \delta_{i,j} - \delta_{i,j+1}\right)\vec{\lambda}_i
\end{equation}

\begin{equation}
	\vec{A}_j - \vec{A}_{j+1} =  \sum_{i=1}^M \left( \delta_{i,j} + \delta_{i,j+2}\right)\vec{\lambda}_i
\end{equation}
leading to the $M$ relations for the $M$ Laplace mutlipliers:
\begin{equation}
	toto
\end{equation}


\end{document}