\documentclass[aps,12pt]{revtex4}
\usepackage{graphicx}
\usepackage{amssymb,amsfonts,amsmath,amsthm}
\usepackage{chemarr}
\usepackage{bm}
\usepackage{pslatex}
\usepackage{bookman}



\begin{document}
\title{Minimal $\mathcal{C}^2$ arcs}
\maketitle

\section{Definitions}
We have a set of $N$ points $\vec{P}_1,\ldots,\vec{P}_N$, and we assume
that we are able to compute the respective velocities  $\vec{V}_1,\ldots,\vec{V}_N$.\\
To produce a $\mathcal{C}^2$ arc from these data, 
me must assign a field of accelerations $\vec{A}_1,\ldots,\vec{A}_N$. ,which imposes and extraneous degree of freedom for each segment.

\section{Accelerations}
On a segment $\vec{P}_i \to \vec{P}_{i+1}$ we get:
\begin{equation}
	\vec{A}(t) = \vec{A}_{i} \left(1-t\right) + t \vec{A}_{i+1} + t\left(1-t\right) \vec{Q}_i
\end{equation}

\section{Velocities}
On a segment $\vec{P}_i \to \vec{P}_{i+1}$ we get:
\begin{equation}
	\vec{V}(t) = \vec{V}_{i} + \int_0^t \vec{A}(u)\,\mathrm{d}u = 
	\vec{V}_{i} + 
	\left(t-\frac{t^2}{2}\right) \vec{A}_{i} 
	+ \frac{t^2}{2} \vec{A}_{i+1}
	+ \left(\frac{t^2}{2} - \frac{t^3}{3}\right) \vec{Q}_i,
\end{equation}
so that:
\begin{equation}
	\vec{V}_{i+1} = \vec{V}_{i} + \frac{1}{2} \left(\vec{A}_i+\vec{A}_{i+1}\right) + \frac{1}{6} \vec{Q}_i.
\end{equation}
Finally, the quadratic term is:
\begin{equation}
\boxed{
	\vec{Q}_i = 6\left(\vec{V}_{i+1}-\vec{V}_{i}\right) - 3  \left(\vec{A}_i+\vec{A}_{i+1}\right) 
}
\end{equation}

\section{Position}
We obtain:
\begin{equation}
	\vec{P}(t) = \vec{P}_{i} + \int_0^t \vec{V}(u)  \,\mathrm{d}u = 
	\vec{P}_i + t \vec{V}_i + \left(\frac{t^2}{2}-\frac{t^3}{6}\right) \vec{A}_{i} 
	+ \frac{t^3}{6} \vec{A}_{i+1}
	+ \left(\frac{t^3}{6} - \frac{t^4}{12}\right) \vec{Q}_i.
\end{equation}
The continuity writes:
\begin{equation}
	\vec{P}_{i+1} = \vec{P}_{i} + \vec{V}_i + \frac{1}{3} \vec{A}_i + \frac{1}{6} \vec{A}_{i+1} + \frac{1}{12} \vec{Q}_i
\end{equation}
then:
\begin{equation}
\label{eq:a}
\boxed{
	\vec{P}_{i+1} - \vec{P}_{i} = \frac{1}{2} \left( \vec{V}_i + \vec{V}_{i+1} \right)
	+\frac{1}{12}\left(\vec{A}_{i}-\vec{A}_{i+1}\right)
	}
\end{equation}
This means that  any set of accelerations matching Eq\eqref{eq:a} shall provide a piecewise $\mathcal{C}^2$ arc.
\section{Expressions}
\subsection{Minimal Arc}
We want to minimize:
\begin{equation}
	\mathcal{E}  = \sum_{i=1}^{N} \frac{1}{2} \vec{A}_i^2,
\end{equation}
under the $M$ constraints:
\begin{equation}
\forall j \in [1:M],\;\;
\left\lbrace
\begin{array}{l}
	  j'=j+1  \mbox{ if } j<N\\
	  j' = 1  \mbox{ if } j=N\\
	 \vec{0}  = \underbrace{\left(\vec{A}_{j}-\vec{A}_{j'}\right) + 6 \left( \vec{V}_j + \vec{V}_{j'} \right ) - 12\left( \vec{P}_{j'} - \vec{P}_{j} \right)}_{\vec{S}_j}\\
\end{array}
\right.
\end{equation}
We define a set $\vec{\lambda}_1,\ldots,\vec{\lambda}_M$ of Lagrange multipliers to define the Lagrangian:
\begin{equation}
	\mathcal{L} = \mathcal{E} - \sum_{i=j}^M \vec{\lambda}_j \vec{S}_j.
\end{equation}

\subsection{Accelerations as a function of Lagrange Multipliers}
We get the expression:
\begin{equation}
	\forall i \in [1:N],\;\; \partial_{\vec{A}_i} \mathcal{L} = \vec{A}_i 
	- \sum_{j=1}^M \left( \delta_{i,j} - \delta_{i,j+1}\right)\vec{\lambda}_j
\end{equation}
leading to the expression of the acceleration in terms of Lagrange multipliers:
\begin{equation}
\boxed{
	\forall k \in [1:N],\;\;\vec{A}_k = \sum_{j=1}^M \left( \delta_{k,j} - \delta_{k,j'}\right)\vec{\lambda}_j
}
\end{equation}

\subsection{Solving The Lagrange Multiplers}
\begin{equation}
\left\lbrace
\begin{array}{rcl}
	\forall k \in [1:M],\;\; \vec{A}_k - \vec{A}_{k'} & = & \displaystyle \sum_{j=1}^M \left[\left( \delta_{k,j} - \delta_{k,j'}\right) - \left( \delta_{k',j} - \delta_{k',j'}\right)\right]\vec{\lambda}_j \\
	\\
	 & = & \underbrace{12\left(\vec{P}_k-\vec{P}_{k'}\right) - 6 \left(\vec{V}_k+\vec{V}_{k'}\right)}_{\vec{D}_k}\\
\end{array}	
\right. 
\end{equation}


\end{document}