\documentclass[aps,12pt]{revtex4}
\usepackage{graphicx}
\usepackage{amssymb,amsfonts,amsmath,amsthm}
\usepackage{chemarr}
\usepackage{bm}
\usepackage{pslatex}
\usepackage{bookman}



\begin{document}
\title{Arcs}
\maketitle
	
\section{Definitions}

\subsection{Positions}
We have a set of $N$ points $\vec{P}_1,\ldots,\vec{P}_N$, each point in $\mathbb{R}^d$ (where $d$ is the space dimension), and we assume
that they describe a virtual arc for a dimensionless coordinate $t_i=i$.

\subsection{Velocities}

\subsubsection{Bulk Case}

\begin{equation}
	\forall 1<i<N,
	\left\lbrace
	\begin{array}{rcl}
	\vec{V}_i & = & \frac{1}{2}\left(\vec{P}_{i+1} - \vec{P}_{i-1}\right)\\
	\vec{A}_i & = &  \vec{P}_{i+1} - \vec{P}_{i-1} + 2\vec{P}_i\\
	\end{array}
	\right.
\end{equation}

\subsubsection{Standard Case Boundaries}
\begin{equation}
\begin{array}{rl}
N=2 &
	\left\lbrace
	\begin{array}{rcl}
	\vec{V}_1 & = &  \left( \vec{P}_2 - \vec{P}_1\right)\\
	\vec{V}_2 & = &  \left( \vec{P}_2 - \vec{P}_1\right) = \vec{V}_1\\
	\vec{A}_1 & = & \vec{0}\\
	\vec{A}_2 & = & \vec{0}\\
	\end{array}
	\right.
	\\
	\\
	N\geq3 & 
	\left\lbrace
	\begin{array}{rcl}
	\vec{V}_1 & = & \frac{1}{2} \left[ 4\vec{P}_2     - (3\vec{P}_1+\vec{P}_3)\right]\\
	\vec{V}_N & = & \frac{1}{2} \left[ (3\vec{P}_{N}+\vec{P}_{N-2})-4\vec{P}_{N-1} \right]\\
	\vec{A}_1 & = & \vec{P}_3+\vec{P}_1-2\vec{P}_2         \\
	\vec{A}_N & = & \vec{P}_N+\vec{P}_{N-2}-2\vec{P}_{N-1} \\
	\end{array}
	\right.
\end{array}
\end{equation}

 
\subsubsection{Periodic Case Boundaries}

\begin{equation}
\begin{array}{rl}
N=2 &
	\left\lbrace
	\begin{array}{rcl}
	\vec{V}_1 & = &  \left( \vec{P}_2 - \vec{P}_1\right)\\
	\vec{V}_2 & = &  \left( \vec{P}_1 - \vec{P}_2\right) = -\vec{V}_1\\
	\vec{A}_1 & = & \vec{0}\\
	\vec{A}_2 & = & \vec{0}\\
	\end{array}
	\right.
	\\
	\\
	N\geq3 & 
	\left\lbrace
	\begin{array}{rcl}
	\vec{V}_1 & = & \frac{1}{2} \left( \vec{P}_2 - \vec{P}_N\right)\\
	\vec{V}_N & = & \frac{1}{2} \left( \vec{P}_1 - \vec{P}_{N-1}\right)\\
	\vec{A}_1 & = & \vec{P}_2+\vec{P}_{N}-2\vec{P}_1         \\
	\vec{A}_N & = & \vec{P}_1+\vec{P}_{N-1}-2\vec{P}_{N} \\
	\end{array}
	\right.
\end{array}
\end{equation}

\section{$\mathcal{C}^0$ Arcs}

\end{document}


\subsection{Generic Index}
We define:
\begin{equation}
	i' =
	\left\lbrace
	\begin{array}{rcl}
		i+1 & \mbox{if } i<N\\
		1   & \mbox{if } i=N\\
	\end{array}
	\right.
\end{equation}


\section{Continuous accelerations}

\subsection{Expression}
We need a degree of freedom to express the acceleration, otherwise we shall get some continuity incompatibility.
Namely, we assume that
\begin{equation}
	\vec{A}(t) = (1-t) \vec{A}_i + t \vec{A}_{i'} + t(1-t) \vec{Q}_i
\end{equation}

\subsection{Velocities}
\begin{equation}
	\vec{V}(t) = \vec{V}_i + \int_0^t \vec{A}(u)\,\mathrm{d}u =
	 \vec{V}_i 
	 + \left(t-\frac{t^2}{2}\right) \vec{A}_i 
	 + \frac{t^2}{2} \vec{A}_{i'}
	 + \left(\frac{t^2}{2}-\frac{t^3}{3}\right) \vec{Q}_i,
\end{equation}
\begin{equation}
	\left(\vec{V}(t) = 
	 \vec{V}_i 
	 + t \left[ \left(1-\frac{t}{2}\right) \vec{A}_i 
	 + \frac{t}{2} \vec{A}_{i'}
	 + \frac{t}{2}\left(1-\frac{2t^2}{3}\right) \vec{Q}_i \right]
	 \right)
\end{equation}

leading to:
\begin{equation}
\boxed{
	\vec{Q}_i = 6 \left( \vec{V}_{i'} - \vec{V}_i \right) - 3 \left( \vec{A}_i + \vec{A}_{i'} \right)
}
\end{equation}

\subsection{Velocities}
\begin{equation}
	\vec{P}(t) = \vec{P}_i + \int_0^t \vec{V}(u) \, \mathrm{d}u = 
	\vec{P}_i 
	+ t \vec{V}_i
	+ \left( \frac{t^2}{2} - \frac{t^3}{6}\right) \vec{A}_i
	+ \frac{t^3}{6} \vec{A}_{i'}
	+ \left(\frac{t^3}{6} -\frac{t^4}{12} \right) \vec{Q}_i,
\end{equation}
\begin{equation}
	\left( \vec{P}(t) = \vec{P}_i 	
	+ t \left[\vec{V}_i
	+ \frac{t}{2}\left[ \left( 1 - \frac{t}{3}\right) \vec{A}_i
	+ \frac{t}{3} \vec{A}_{i'}
	+ \frac{t}{3}\left(1 -\frac{t}{2} \right) \vec{Q}_i
	\right]
	\right]
	 \right)
\end{equation}
leading to
\begin{equation}
	\vec{P}_{i'} =  \vec{P}_{i} + \vec{V}_i + \frac{1}{3} \vec{A}_i + \frac{1}{6} \vec{A}_{i'} + \frac{1}{12} \vec{Q}_i
\end{equation}
and
\begin{equation}
\label{eq:A}
\boxed{
	\vec{P}_{i'} - \vec{P}_i = \frac{1}{2}\left( \vec{V}_i + \vec{V}_i' \right) + \frac{1}{12} \left( \vec{A}_i - \vec{A}_{i'} \right)
	}
\end{equation}

\section{Minimal Arc}

\subsection{Choosing Accelerations}
Any set of accelerations matching Eq\eqref{eq:A} shall produce an arc going through all points with the computed velocities.
Accordingly, we choose to minimise:
\begin{equation}
	\mathcal{E} = \frac{1}{2} \sum_{i=0}^N \vec{A}_i^2
\end{equation}
with the $N-1$ constraints:
\begin{equation}
	\vec{0} = \underbrace{\left(\vec{A}_i  - \vec{A}_{i'} \right) + \underbrace{6 \left(\vec{V}_{i} + \vec{V}_{i'}\right) - 12 \left(\vec{P}_{i'} - \vec{P}_{i}\right)}_{-\vec{D}_i}}_{\vec{C}_i}.
\end{equation}
We note that this works as well for the periodic case, thanks to the velocities expression.\\
We introduce a set of $N-1$ Lagrange multipliers $\vec{\lambda}_j$ (or $d(N-1)$ scalars) and we introduce the Lagrangian:
\begin{equation}
	\mathcal{L} = \mathcal{E} - \sum_{j=1}^{j<N} \vec{\lambda_j} \vec{C}_j	
\end{equation}
\subsection{Expression}
Accordingly, we express the accelerations as a function of the Lagrange multipliers:
\begin{equation}
	\forall 1\leq k \leq N, \;\; \vec{A}_k = \sum_{j=1}^{j<N} \left( \delta_{k,j} - \delta_{k,j'}\right) \vec{\lambda}_j,
\end{equation}
and consequently:
\begin{equation}
	\forall 1\leq k < N, \;\; \vec{A}_k - \vec{A}_{k'}
	= \sum_{j=1}^{j<N} \left[ 
	\left(\delta_{k,j}+\delta_{k',j'} \right) 
	- \left(\delta_{k,j'}+\delta_{k',j}\right)
	\right] \vec{\lambda}_j
\end{equation}
We then have the algebraic solution for the multipliers:
\begin{equation}
	\forall 1\leq k < N,\;\;\sum_{j=1}^{j<N} \left[ 
	\left(\delta_{k,j}+\delta_{k',j'} \right) 
	- \left(\delta_{k,j'}+\delta_{k',j}\right)
	\right] \vec{\lambda}_j = \vec{D}_k
\end{equation}

\end{document}


\end{document}