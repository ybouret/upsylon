\documentclass[aps]{revtex4}
\usepackage{graphicx}
\usepackage{amssymb,amsfonts,amsmath,amsthm}
\usepackage{chemarr}
\usepackage{bm}
\usepackage{pslatex}
\usepackage{mathptmx}
\usepackage{xfrac}

%% concentration notations
\newcommand{\mymat}[1]{\boldsymbol{#1}}
\newcommand{\mytrn}[1]{~^{\mathsf{t}}\!{#1}}
\newcommand{\myvec}[1]{\overrightarrow{#1}}
\newcommand{\mygrad}{\vec{\nabla}}
\newcommand{\myhess}{\mathcal{H}}


\begin{document}
\title{Least Squares}
\maketitle
	
\section{Notations}

\subsection{Standard Problem}
Let us assume that we have a set of points $\lbrace x_i, y_i \rbrace_{i\in[1;N]}$ that we want to adjust by
the objective function $F(x,\vec{a})$, by minimising the \textbf{sum of squares}:
\begin{equation}
	D^2(\vec{a}) = \frac{1}{2} \sum_{i=1}^{N}  \left[ y_i - \underbrace{F(x_i,\vec{a})}_{F_i} \right] ^2 
\end{equation}
The \textbf{descent direction} is:
\begin{equation}
	\vec{\beta} = -\mygrad D^2 = -\partial_{\vec{a}} D^2
\end{equation}
with:
\begin{equation}
	\beta_j = \sum_{i=1}^{N} \left[y_i - F_i\right] \left(\dfrac{\partial F_i}{\partial a_j}\right)
\end{equation}
Since an extremum of $D^2$ is found when $\vec{\beta}=\vec{0}$, we want to look for null values of the
descent direction. We expand:
\begin{equation}
	\vec{\beta}\left(\vec{a}+\delta\vec{a}\right) = \vec{\beta}\left(\vec{a}\right) + \partial_{\vec{a}}\vec{\beta} \cdot \delta\vec{a}
\end{equation}
and
\begin{equation}
	\dfrac{\partial \beta_j}{\partial a_k} =  \left[\sum_{i=1}^{N} \left[y_i - F_i\right] \left(\dfrac{\partial^2 F_i}{\partial a_j\partial a_k}\right) \right] - 
	\underbrace{\left[ \sum_{i=1}^N \left(\dfrac{\partial F_i}{\partial a_j}\right) \left(\dfrac{\partial F_i}{\partial a_k}\right)\right]}_{\alpha_{ij}}
\end{equation}
We approximate the Hessian matrix by $\mymat{\alpha}_{\lambda}$ defined by the multiplication of the diagonal terms of $\mymat{\alpha}$ by $1+\lambda$.
Hence, we get the \textbf{full quadratic step}:

\begin{equation}
	\delta\vec{a}_\lambda = \mymat{\alpha}_\lambda^{-1} \vec{\beta}
\end{equation}


\section{Algorithm}


\end{document}


