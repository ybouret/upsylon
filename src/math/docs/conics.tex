\documentclass[aps]{revtex4}
\usepackage{graphicx}
\usepackage{amssymb,amsfonts,amsmath,amsthm}
\usepackage{chemarr}
\usepackage{bm}
\usepackage{pslatex}
\usepackage{mathptmx}
\usepackage{xfrac}
\usepackage{dejavu}

\newcommand{\mymat}[1]{\boldsymbol{#1}}
%\newcommand{\mytrn}[1]{{#1}^{\mathsf{T}}}
\newcommand{\mytrn}[1]{~^{\mathsf{t}}\!{#1}}
\newcommand{\myvec}[1]{\overrightarrow{#1}}
\newcommand{\mygrad}{\vec{\nabla}}
\newcommand{\myhess}{\mathcal{H}}


\begin{document}
\title{Fitting to conics}
\maketitle
\section{Settings}

We have a set of $N$ points $\vec{M}_i = (x_i,y_i)$ in the plane.
A generic conic equation is given by:
\begin{equation}
	a x^2 + b y^2 + c xy + d x + e y + f = \vec{A}\cdot\vec{Z} = 0
\end{equation}
with
\begin{equation}
	\vec{A} = \begin{bmatrix}
		a\\
		b\\
		c\\
		d\\
		e\\
		f\\
	\end{bmatrix}, \;\;
	\vec{Z} = \begin{bmatrix}
	x^2\\
	y^2\\
	xy\\
	x\\
	y\\
	1\\
	\end{bmatrix}
\end{equation}

To fit a conic equation to a set of points, we have to minimise:
\begin{equation}
	F(\vec{A}) = \sum_{i=1}^N \left( \vec{A}\cdot\vec{Z}_i\right)^2 =
	 \mytrn{\vec{A}} \underbrace{\left(\sum_{i=1}^N \vec{Z}_i\mytrn{\vec{Z}}_i\right)}_{\mymat{S}} \vec{A}
\end{equation}

\end{document}
