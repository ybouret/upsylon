\documentclass[aps,10pt]{revtex4}
\usepackage{graphicx}
\usepackage{amssymb,amsfonts,amsmath,amsthm}
\usepackage{chemarr}
\usepackage{bm}
\usepackage{pslatex}
\usepackage{bookman}

\begin{document}
\title{Splines for curves 1D/2D/3D/...}
\maketitle

\section{Definitions}
We have a set of $N$ points $\vec{P}_1,\ldots,\vec{P}_N$ at different times $t_1,\ldots,t_N$.
On each interval, we define the formal terms:
\begin{equation}
\left\lbrace
\begin{array}{rcl}
	A_i & = &\dfrac{t_{i+1}-t}{t_{i+1}-t_{i}}\\
	\\
	B_i & = &\dfrac{t-t_{i}}{t_{i+1}-t{i}} = 1-A_i\\
	\\
	C_i & = & \dfrac{1}{6}\left(A_i^3-A_i\right)\left(t_{i+1}-t_{i}\right)^2\\
	\\
	D_i & = & \dfrac{1}{6}\left(B_i^3-B_i\right)\left(t_{i+1}-t_{i}\right)^2\\
\end{array}
\right.
\end{equation}
so that we have  $N-1$ spline approximations:
\begin{equation}
	\forall t \in [t_{i};t_{i+1}], \;\; \vec{S}_i = A_i \vec{P}_{i} + B_i \vec{P}_{i+1} + C_i \vec{Q}_i + D_i \vec{Q}_{i+1}
\end{equation}
and a speed approximation:
\begin{equation}	
		\forall t \in [t_{i};t_{i+1}],
		 \;\; \vec{S}'_i(t) = 
		 \dfrac{1}{t_{i+1}-t_{i}} \left(\vec{P}_{i+1}-\vec{P}_i\right)
		  - \dfrac{3A_i^2-1}{6}(t_{i+1}-t_{i}) \vec{Q}_i
		  + \dfrac{3B_i^2-1}{6}(t_{i+1}-t_{i}) \vec{Q}_{i+1}
\end{equation}
and the boundary values
\begin{equation}
\left\lbrace
\begin{array}{rcl}
	\vec{S}'_1(0) & = & \dfrac{1}{t_{2}-t_{1}} \left(\vec{P}_{2}-\vec{P}_1\right)
		  - \dfrac{1}{3}(t_{2}-t_{1}) \vec{Q}_1
		  - \dfrac{1}{6}(t_{2}-t_{1}) \vec{Q}_{2}\\
		  \\
		 \vec{S}'_{N-1}(t_N) & = &  \dfrac{1}{t_{N}-t_{N-1}} \left(\vec{P}_{N}-\vec{P}_{N-1}\right)  + \dfrac{1}{6}(t_{N}-t_{N-1}) \vec{Q}_{N-1}
		  + \dfrac{1}{3}(t_{N}-t_{N-1}) \vec{Q}_{N} \\
\end{array}
\right.
\end{equation}

We have as well:
\begin{equation}	
		\forall t \in [t_{i};t_{i+1}],
		 \;\; \vec{S}''_i(t) = A_i \vec{Q}_i + B_i \vec{Q}_{i+1}
\end{equation}
We have $N$ values $\vec{Q}_1,\ldots,\vec{Q}_N$ to compute.
The speed continuity yields:

\begin{equation}
\label{eq:bulk}
\begin{array}{rl}
 &\forall i \in [2;N-1],\;\; \vec{S}_{i-1}(t_{i}) = \vec{S}_{i}(t_i)\\
\Leftrightarrow &  
 \dfrac{1}{t_{i}-t_{i-1}} \left(\vec{P}_{i}-\vec{P}_{i-1}\right)
		  + \dfrac{1}{6}(t_{i}-t_{i-1}) \vec{Q}_{i-1}
		  + \dfrac{1}{3}(t_{i}-t_{i-1}) \vec{Q}_{i}\\	  
 & = \dfrac{1}{t_{i+1}-t_{i}} \left(\vec{P}_{i+1}-\vec{P}_i\right)
		  - \dfrac{1}{3}(t_{i+1}-t_{i}) \vec{Q}_i
		  - \dfrac{1}{6}(t_{i+1}-t_{i}) \vec{Q}_{i+1}  \\
		  \\
\Leftrightarrow & \dfrac{1}{t_{i+1}-t_{i}} \left(\vec{P}_{i+1}-\vec{P}_i\right) - \dfrac{1}{t_{i}-t_{i-1}} \left(\vec{P}_{i}-\vec{P}_{i-1}\right)
= \dfrac{1}{6}(t_{i}-t_{i-1}) \vec{Q}_{i-1} + \dfrac{1}{3}(t_{i+1}-t_{i-1}) \vec{Q}_{i} + \dfrac{1}{6}(t_{i+1}-t_{i}) \vec{Q}_{i+1} 
\\
\end{array}
\end{equation}
and we have $N-2$ tridiagonal equations that shall be completed by 2 boundary conditions.

\section{Boundary conditions}

\subsection{Standard Splines}
\begin{itemize}
\item The natural condition: 
\begin{equation}
\left\lbrace
\begin{array}{rcl}
\vec{Q}_1 & = &\vec{0} \\
\text{or/and}\\
\vec{Q}_2 & = & \vec{0}\\
\end{array}
\right.
\end{equation}

\item The fixed derivative condition:
\begin{equation}
\left\lbrace
\begin{array}{rcl}
	\vec{Q}_1 + \dfrac{1}{2}\vec{Q}_2 & = &\dfrac{3}{t_2-t_1} \left[ \dfrac{1}{t_2-t_1} \left(\vec{P}_2-\vec{P}_1\right) - \partial_t \vec{P}_1 \right]\\
	\text{or/and}\\
	\dfrac{1}{2}\vec{Q}_{N-1} +\vec{Q}_N& =& \dfrac{3}{t_N-t_{N-1}} \left[\partial_t \vec{P}_{N} - \dfrac{1}{t_N-t_{N-1}}\left(\vec{P}_N - \vec{P}_{N-1}\right)\right]\\
\end{array}
\right.
\end{equation}
\end{itemize}

\subsection{Periodic Splines}
We still have the $N-2$ conditions defined by Eq \eqref{eq:bulk} and we also get
\begin{equation}
\left\lbrace
\begin{array}{rcl}
\dfrac{1}{t_{N+1}-t_{N}} \left(\vec{P}_{1}-\vec{P}_N\right) - \dfrac{1}{t_{N}-t_{N-1}} \left(\vec{P}_{N}-\vec{P}_{N-1}\right)
 &= & \dfrac{1}{6}(t_{N}-t_{N-1}) \vec{Q}_{N-1} + \dfrac{1}{3}(t_{N+1}-t_{N-1}) \vec{Q}_{N} + \dfrac{1}{6}(t_{N+1}-t_{i}) \vec{Q}_{1}\\
\\
\dfrac{1}{t_{2}-t_{1}} \left(\vec{P}_{2}-\vec{P}_1\right) - \dfrac{1}{t_{1}-t_{0}} \left(\vec{P}_{1}-\vec{P}_{N}\right)
 &=& \dfrac{1}{6}(t_{1}-t_{0}) \vec{Q}_{N} + \dfrac{1}{3}(t_{2}-t_{0}) \vec{Q}_{1} + \dfrac{1}{6}(t_{2}-t_{1}) \vec{Q}_{2} \\ 
\end{array}
\right.
\end{equation}
where $t_0$ and $t_{N+1}$ are choosen to match the periodic boundary conditions.

\section{Virtual Spline, $t_i=i$}

\subsection{Standard Spline}
The main $N-2$ relations become:
\begin{equation}
	\forall i \in[2;N-1],\;\;6\left[\vec{P}_{i+1} - 2 \vec{P}_i + \vec{P}_{i-1}\right] =  \vec{Q}_{i-1} + 4 \vec{Q}_i + \vec{Q}_{i+1}
\end{equation}
And we can fix:
\begin{equation}
\left\lbrace
\begin{array}{rcl}
\vec{Q}_1 & = & \vec{0} \\
\\
\text{ or }\\
\\
\vec{Q}_1 + \dfrac{1}{2}\vec{Q}_2 & = & 3\left[\left(\vec{P}_2-\vec{P}_1\right) - \partial_t \vec{P}_1\right]\\
\end{array}
\right.
\end{equation}
and:
\begin{equation}
\left\lbrace
\begin{array}{rcl}
\vec{Q}_N & = & \vec{0} \\
\\
\text{ or }\\
\\
\vec{Q}_N + \dfrac{1}{2}\vec{Q}_{N-1} & = & 3\left[\partial_t \vec{P}_{N} - \left(\vec{P}_N-\vec{P}_{N-1}\right)  \right]\\
\end{array}
\right.
\end{equation}
Hence, the derivatives $\partial_t \vec{P}_1$ and $\partial_t \vec{P}_N$ are also virtual, and should be chosen to adjust the initial or final metrics.

\subsection{Periodic Spline}
We obtain the periodic system:
\begin{equation}
\left\lbrace
\begin{array}{lccl}
 & \vec{Q}_{N} + 4 \vec{Q}_1 + \vec{Q}_{2} & = & 6\left[\vec{P}_{2} - 2 \vec{P}_1 + \vec{P}_{N}\right] \\
 \\
 \forall i \in[2;N-1], & \vec{Q}_{i-1} + 4 \vec{Q}_i + \vec{Q}_{i+1} & = & 6\left[\vec{P}_{i+1} - 2 \vec{P}_i + \vec{P}_{i-1}\right] \\
 \\
 & \vec{Q}_{N-1} + 4 \vec{Q}_N + \vec{Q}_{1} & = & 6\left[\vec{P}_{1} - 2 \vec{P}_N + \vec{P}_{N-1}\right] \\
\end{array}
\right.
\end{equation}

\end{document}