\documentclass[aps,12pt]{revtex4}
\usepackage{graphicx}
\usepackage{amssymb,amsfonts,amsmath,amsthm}
\usepackage{chemarr}
\usepackage{bm}
\usepackage{pslatex}
\usepackage{mathptmx}
\usepackage{xfrac}
\usepackage{dejavu}
\usepackage{dsfont}


\newcommand{\grad}{\vec{\nabla}}
\newcommand{\divg}{{\mathrm{div}\,}}
\newcommand{\rot}{\overrightarrow{\mathrm{rot}}\,}

\begin{document}

Let us consider a species within a chemical potential:
\begin{equation}
	\mu_i(\vec{r}) = \mu_i^0(T) + RT \ln C_i(\vec{r}) +  z_i \mathcal{F} V(\vec{r})
\end{equation}

\begin{equation}
	\vec{E} = -\grad V
\end{equation}

\begin{equation}
	\Delta V = -\dfrac{\rho}{\epsilon}
\end{equation}

The isothermal molar force field is:
\begin{equation}
	\vec{F}_i(\vec{r}) = -\grad \mu_{i} = -\frac{RT}{C_i} \grad C_i + z_i \mathcal{F} \vec{E}
\end{equation}
One particle with mass $m_i$ and a friction coefficient $\eta_i$ moves with:
\begin{equation}
	m_i  \ddot{\vec{OM}} = -\eta_i  \dot{\vec{OM}} + \dfrac{1}{\mathcal{N}_A} \vec{F}_i
\end{equation}
The relaxed velocity is:
\begin{equation}
	\vec{v}_i = \dfrac{1}{\eta_i\mathcal{N}_A} \vec{F}_i = -\dfrac{kT}{\eta_i C_i} \grad C_i + \frac{z_i e}{\eta_i} \vec{E}
\end{equation}
The flux is:
\begin{equation}
	\vec{J}_i = C_i \vec{v}_i = -\dfrac{kT}{\eta_i} \grad C_i + \dfrac{z_i e}{\eta_i} C_i \vec{E}.
\end{equation}
The electric flux is:
\begin{equation}
	\vec{J}^{el}_i = z_i e \vec{J_i} = -\dfrac{ez_ikT}{\eta_i} \grad C_i + \dfrac{z_i^2 e^2}{\eta_i} C_i \vec{E}.
\end{equation}
We use
\begin{equation}
	D_i = \dfrac{kT}{\eta_i}, 
\end{equation}

\begin{equation}
\left\lbrace
\begin{array}{rcl}
\vec{J}_i & = &  -D_i \grad C_i + C_i \dfrac{z_i D_i}{\Psi_0}  \vec{E} \;\; (\Psi_0=\dfrac{kT}{e}\simeq 25.7 mV)\\
\\
\vec{J}^{el}_i & = & - z_i D_i e  \grad C_i + C_i \dfrac{z_i^2 D_i e}{\Psi_0}  \vec{E} \\
\end{array}
\right.
\end{equation}


The matter conservation writes:
\begin{equation}
	\divg \vec{J}_i + \partial_t C_i = 0
\end{equation}
The charge conservation writes:
\begin{equation}
	\divg \left( \sum_i \vec{J}^{el}_i \right) + \partial_t \rho = 0, \;\; \rho = e \sum_i z_i C_i
\end{equation}

\begin{quote}
\centerline{$<$Vectorial Analysis$>$}
\begin{equation}
	\divg \vec{F} = \vec{0} \Leftrightarrow \vec{F} = \rot (\vec{A} + \grad f)
\end{equation}

\begin{equation}
	\divg (a\vec{b}) = a \cdot \divg \vec{b} + \grad a \cdot \vec{b}
\end{equation}
\centerline{$<$Vectorial Analysis$/>$}

\end{quote}

To conserve the local charge,
\begin{equation}
	\sum_i \vec{J}^{el}_i = \sum_i\left( - z_i D_i e  \grad C_i + C_i \dfrac{z_i^2 D_i e}{\Psi_0}  \vec{E} \right) = e \; \rot\vec{w}
\end{equation}

\begin{equation}
	\vec{E} = \Psi_0 \left[ \sum_i z_i^2 D_i C_i  \right]^{-1} \left[ \rot\vec{w} + \sum_i z_i D_i \grad C_i \right]
\end{equation}

\end{document}
